%==============================================================================
% Sjabloon onderzoeksvoorstel bachelorproef
%==============================================================================
% Gebaseerd op LaTeX-sjabloon ‘Stylish Article’ (zie voorstel.cls)
% Auteur: Jens Buysse, Bert Van Vreckem
%
% Compileren in TeXstudio:
%
% - Zorg dat Biber de bibliografie compileert (en niet Biblatex)
%   Options > Configure > Build > Default Bibliography Tool: "txs:///biber"
% - F5 om te compileren en het resultaat te bekijken.
% - Als de bibliografie niet zichtbaar is, probeer dan F5 - F8 - F5
%   Met F8 compileer je de bibliografie apart.
%
% Als je JabRef gebruikt voor het bijhouden van de bibliografie, zorg dan
% dat je in ``biblatex''-modus opslaat: File > Switch to BibLaTeX mode.

\documentclass{voorstel}

\usepackage{lipsum}

%------------------------------------------------------------------------------
% Metadata over het voorstel
%------------------------------------------------------------------------------

%---------- Titel & auteur ----------------------------------------------------

% TODO: geef werktitel van je eigen voorstel op
\PaperTitle{De interactie van een progressive web application met het besturingssysteem: een vergelijkende studie en proof-of-concept}
\PaperType{Onderzoeksvoorstel Bachelorproef 2019-2020} % Type document

% TODO: vul je eigen naam in als auteur, geef ook je emailadres mee!
\Authors{Tijs Martens\textsuperscript{1}} % Authors
\CoPromotor{Simon Floré\textsuperscript{2} (Bothrs)}
\affiliation{\textbf{Contact:}
  \textsuperscript{1} \href{mailto:tijs.martens@student.hogent.be}{tijs.martens@student.hogent.be};
  \textsuperscript{2} \href{mailto:simon@bothrs.com}{simon@bothrs.com};
}

%---------- Abstract ----------------------------------------------------------

\Abstract{
    Er zal onderzoek gedaan worden naar de huidige staat van progressive web applications (PWA’s) en hoe PWA's gebruik kunnen maken van de functies van een besturingssysteem.
    
    In een eerste deel van het onderzoek zal onderzocht worden wat een PWA juist is en hoe dit gemaakt kan worden.
    Vervolgens zal er een vergelijkende studie uitgevoerd worden tussen PWA's en native applications. Hierbij zal er nagegaan worden welke door het besturingssysteem aangeboden functies kunnen gebruikt worden door een PWA.
    In het derde deel van het onderzoek zal er gekeken worden hoe de verschillende mobiele besturingssystemen staan ten opzichte van PWA's.
    Uiteindelijk zal er ook gezocht worden naar eventuele alternatieven voor PWA's.
   
    
    Dit onderzoek is relevant omdat het ontwikkelen van PWA’s een stuk sneller en goedkoper is dan het ontwikkelen van native applications. Door het in kaart brengen van wat mogelijk is en wat niet, kan er beter beslist worden wanneer er een PWA kan gebruikt worden in plaats van een native application. Op deze manier kan er voor sommige bedrijven tijd en budget gespaard worden zonder kwaliteit in te leveren.
    De verwachting is dat voor veel use-cases een PWA volstaat en er geen ontwikkeling van native applications voor verschillende platformen nodig is.
    
    Simon Floré van experience design studio Bothrs zal me bijstaan in dit onderzoek.
    
    
}

%---------- Onderzoeksdomein en sleutelwoorden --------------------------------
% TODO: Sleutelwoorden:
%
% Het eerste sleutelwoord beschrijft het onderzoeksdomein. Je kan kiezen uit
% deze lijst:
%
% - Mobiele applicatieontwikkeling
% - Webapplicatieontwikkeling
% - Applicatieontwikkeling (andere)
% - Systeembeheer
% - Netwerkbeheer
% - Mainframe
% - E-business
% - Databanken en big data
% - Machineleertechnieken en kunstmatige intelligentie
% - Andere (specifieer)
%
% De andere sleutelwoorden zijn vrij te kiezen

\Keywords{Webapplicatieontwikkeling. PWA --- native apps --- besturingssystemen} % Keywords
\newcommand{\keywordname}{Sleutelwoorden} % Defines the keywords heading name

%---------- Titel, inhoud -----------------------------------------------------

\begin{document}
    
\flushbottom % Makes all text pages the same height
\maketitle % Print the title and abstract box
        \tableofcontents % Print the contents section

\thispagestyle{empty} % Removes page numbering from the first page

%------------------------------------------------------------------------------
% Hoofdtekst
%------------------------------------------------------------------------------

% De hoofdtekst van het voorstel zit in een apart bestand, zodat het makkelijk
% kan opgenomen worden in de bijlagen van de bachelorproef zelf.
%---------- Inleiding ---------------------------------------------------------

\section{Introductie} % The \section*{} command stops section numbering
\label{sec:introductie}

Op een desktop is het gebruikelijk om taken uit te voeren in de browser. Voorbeelden hiervan zijn Gmail en Google drive. Dit is echter nog niet het geval op mobiele toestellen.
Elke digitale toepassing, waarbij mobiele gebruikers het doelpubliek zijn, heeft een native application nodig. Vaak volstaat een traditionele responsive website niet omdat er essentiële functies zoals push notifications ontbreken. Dit heeft als gevolg dat gebruikers minder snel terugkeren naar jouw website.
\autocite{Hiltunen2018}

De ontwikkeling van native applications is echter geen goedkope of snelle oplossing. Er moet gelijkaardige code herschreven worden voor meerdere platformen. Digitale agentschappen, startups en andere software-ontwikkelaars willen vaak een zo snel mogelijke service bieden aan hun klanten. Dit is momenteel moeilijk.

PWA's kunnen voor deze problemen een oplossing bieden.
Bij het ontwikkelen van een mobiele applicatie zijn vaak functies, die aangeboden worden door het besturingssysteem, nodig. Voorbeelden hiervan zijn: locatievoorzieningen, offline gebruik, camera, push notifications, …

In deze thesis zullen volgende onderzoeksvragen uitgewerkt worden:
\begin{itemize}
    \item Welke stappen zijn nodig om een traditionele website om te vormen tot een PWA?
    \item Wat zijn de beperkingen van een PWA?
    \item Kan een PWA alle functionaliteiten gebruiken die beschikbaar zijn voor native applications?
    \item Hoe staan de verschillende besturingssystemen ten opzichte van PWA's?
    \item Welke andere technologieën kunnen er gebruikt worden om applicaties te ontwikkelen voor meerdere platformen waarbij er maar één codebase is?
\end{itemize}

 



%---------- Stand van zaken ---------------------------------------------------

\section{State-of-the-art}
\label{sec:state-of-the-art}
\subsection{Wat is een PWA}

Een PWA is een website gebouwd met web technologieën die zich gedraagt als een native application maar waarbij er niet door het installatieproces moet gegaan worden zoals bij native applications.
\autocite{Sayali2018}

Een PWA is een 'enhanced website', dit is een website met een paar extra bestanden die er voor zorgen dat de site extra functionaliteiten heeft. Volgende bestanden zijn nodig om van een website een PWA te maken:
\begin{itemize}
    \item Manifest.json – Dit is een bestand waar je enkele eigenschappen van de applicatie instelt zoals: app icoon, startpagina, kleurschema, …
    \item Service worker – Dit is een bestand waarbij je zelf caching kan doen. Hierdoor kan, eens een website geladen is, de site offline gebruikt worden.
\end{itemize}
\autocite{Harris2017}

\subsection{Welke functies van een besturingssysteem kan een PWA gebruiken}
Niet alle besturingssystemen stellen evenveel functies beschikbaar die gebruikt kunnen worden vanuit een PWA: IOS and Android stellen volgende functies ter beschikking
\begin{itemize}
    \item Notificaties op het toestel
    \item Locatievoorziening
    \item Camera
    \item Gebruik van de sensoren van het toestel
    \item Audio-output
    \item Betalingssystemen (Android pay voor Android, Apple pay voor IOS)
    \item Spraakinput
    \item Bluetooth (enkel Android)
\end{itemize}
Andere functies waar native applications wel toegang tot hebben, zijn niet beschikbaar voor PWA's:
\begin{itemize}
    \item Toegang tot contacten
    \item Toegang tot de kalender
    \item Toegang tot alarmen
    \item Toegang tot telefoniedata
    \item Berichten
    \item Belfunctie
    \item Toegang tot low level hardware sensoren
    \item Camera (video’s)
    \item Maximum opslag van 50Mb op IOS
    \item Geen widgets
\end{itemize}

\autocite{Malavolta2016}
\autocite{Destrebecq2018}

\subsection{Waarom PWA's}

De verwachtingen die een gebruiker heeft van een website of digitale toepassing zijn hoger dan ooit. Als je website niet binnen 3 seconden geladen is, zal je al 53\% van de gebruikers verliezen. Dit kan voorkomen worden door progressive web applications. Als de website éénmaal geladen is, kan de site opgeslagen worden in het cachegeheugen. Dit zorgt voor een snellere ervaring.
\autocite{Google2017}

Gebruikers raken gefrustreerd als het gedrag van een mobiele applicatie anders is op de verschillende platformen. Dit verschil komt er omdat dit totaal verschillende code-bases zijn, die vaak door verschillende teams ontwikkeld en onderhouden worden. Met PWA's wordt er één codebase geschreven die op alle platformen gebruikt wordt. Hierdoor zal het systeem op elk platform gelijkaardig werken.
\autocite{Google2019}

Volgens Google zijn er drie grote redenen om over te schakelen naar PWA’s:
\begin{itemize}
    \item Betrouwbaarheid – Door het cachegeheugen zal een gebruiker nooit op een pagina terechtkomen met een melding dat er geen internetconnectie is.
    \item Snelheid – Een PWA kan sneller zijn dan een gewone website door het gebruik van cachegeheugen. Een PWA kan sneller zijn dan een native application doordat het een veel kleiner bestand is.
    \item Aantrekkelijkheid – een PWA kan aanvoelen als een native application.
\end{itemize}
\autocite{GooglePwa2019}

\subsection{Beveiliging}
Een PWA moet altijd een HTTPS-verbinding hebben. Dit wil zeggen dat de data die tussen de client en de server verstuurd wordt, versleuteld is.
\autocite{Durumeric2013}
Dit is een stap in de juiste richting maar het zorgt er dus niet voor dat elke PWA een veilige toepassing is. PWA’s worden vandaag gebruikt om gebruikers op te lichten. Een vaak gebruikte techniek is dat een bestaande native application wordt nagemaakt. Op deze manier proberen ze wachtwoorden en betalingsdetails te verkrijgen.
\autocite{Lee2018}

\subsection{Native containers}
Niet alle problemen kunnen opgelost worden via het web de dag van vandaag. Voor sommige taken zijn nog steeds native applications nodig. Als er een PWA moet gemaakt worden met de functies van een native application, kan deze PWA ontsloten worden door een native wrapper.
Een nadeel hiervan is dat de grootte van een PWA drastisch verhoogt.
Een ander nadeel is dat er twee of meer codebases onderhouden moeten worden, één van de site en minstens één van de wrappers.
Deze websites zijn dan niet meer beschikbaar via de browser en moeten geïnstalleerd worden via een app-store.
Dit wordt ook wel ‘web based hybrid mobile app’ genoemd.
\autocite{Richard2019} \autocite{Malavolta2016}

\subsection{Apllication shell architecture}
Om een snelle ervaring te bieden aan de eindgebruiker moet een applicatie opgedeeld worden in twee delen: de inhoud en de ‘application shell’.
De application shell is het deel van de interface die op elke pagina terugkomt. Dit kan bestaan uit achtergronden, navigatie, …
Deze application shell moet volledig lokaal opgeslagen worden zodat deze niet steeds opnieuw gedownload moet worden.
\autocite{Hiltunen2018}


% Voor literatuurverwijzingen zijn er twee belangrijke commando's:
% \autocite{KEY} => (Auteur, jaartal) Gebruik dit als de naam van de auteur
%   geen onderdeel is van de zin.
% \textcite{KEY} => Auteur (jaartal)  Gebruik dit als de auteursnaam wel een
%   functie heeft in de zin (bv. ``Uit onderzoek door Doll & Hill (1954) bleek
%   ...'')


%---------- Methodologie ------------------------------------------------------
\section{Methodologie}
\label{sec:methodologie}

In een eerste fase van het onderzoek wordt bekeken wat er gedaan moet worden om een bestaande traditionele webapplicatie om te vormen tot een PWA. Dit houdt in dat de gebruiker deze website op zijn toestel kan installeren. 

Bij het tweede luik van het onderzoek zal er een vergelijkende studie uitgevoerd worden tussen PWA's en native applications. Er zal een businesscase uitgewerkt worden als PWA en als native application, deze businesscase zal bepaald worden in samenspraak met Bothrs. 
Tijdens deze ontwikkeling zal er gekeken worden welke functies van het besturingssysteem gebruikt kunnen worden door PWA's en welke niet.

Na het voeren van deze twee praktische onderzoeken zullen er ook theoretische onderzoeken gevoerd worden. 
Er zal bekeken worden welke besturingssystemen de meeste functionaliteiten bieden waar PWA's gebruik kunnen van maken. 
Er zal ook in kaart gebracht worden wat de zwaktes en limiterende factoren zijn van een PWA ten opzichte van een native application.
In een laatste deel van het onderzoek zal er gezocht worden naar andere technologieën waarbij applicaties kunnen ontwikkeld worden voor meerdere platformen met één codebase.


\subsection{Technologieën}

PWA's kunnen gemaakt worden met de tools en technologiën die gebruikt worden om traditionele webapplicaties te bouwen. 
Voor de onderzoeken zal voor de eerste fase HTML, CSS en JavaScript gebruikt worden. Eventueel kan er ook gebruik gemaakt worden van een javascript library zoals React.
Er zijn verschillende tools beschikbaar voor het creëren van PWA’s. Eén van deze tools is Ionic.
Voor het maken van een native shell zal er waarschijnlijk gebruik gemaakt worden van react native of flutter. Dit zijn frameworks waarbij één codebase gebruikt wordt om apps te ontwikkelen voor IOS en Android.




%---------- Verwachte resultaten ----------------------------------------------
\section{Verwachte resultaten}
\label{sec:verwachte_resultaten}


Een website maken die voldoet aan de normen van een PWA zal niet veel tijd in beslag nemen. Er moeten slechts twee bestanden toegevoegd worden: het app-manifest bestand en een serviceworker. Het app-manifest bestand bepaalt de look en feel van de applicatie eens deze geïnstalleerd is. 
De serviceworker zorgt voor het cachen van data afkomstig van een API. Echter, om aan de normen van een PWA te voldoen, moet deze file gewoon aanwezig zijn. 
Als dit het geval is, kan de website geïnstalleerd worden op het toestel van de eindgebruiker en voldoet de website dus aan de normen van een PWA.

Het effectief gebruik maken van de functionaliteiten die een serviceworker kan bieden, kent een steilere leercurve. De ontwikkelaar moet verschillende, complexe concepten begrijpen en kunnen toepassen. 
Een voorbeeld hiervan is de levenscyclus die een serviceworker doorloopt. Dit is belangrijk voor het updaten van de inhoud van de website op basis van het cachegeheugen of de API. 
Om een goede offline ervaring te bieden zal het ontwerp van de webapplicatie ook herzien moeten worden. Er moet een application shell gedefinieerd worden.
\autocite{Gaunt2019} \autocite{Osmani2016}

Er wordt verwacht dat er bij de besturingssystemen een duidelijk verschil zal zijn. 
Google, de ontwikkelaar van Android, heeft de revolutie die PWA's wel zijn, gestart.
Hierdoor wordt er verwacht dat Android meer functies van het besturingssysteem open zal stellen voor PWA’s.
Het is daarentegen bekend dat Apple toegang tot het besturingssysteem zoveel mogelijk wil beperken. Een duidelijk verschil tussen deze twee besturingssystemen wordt verwacht. 
\autocite{Hansen2017}




%---------- Verwachte conclusies ----------------------------------------------
\section{Verwachte conclusies}
\label{sec:verwachte_conclusies}

Native applications zullen waarschijnlijk niet snel vervangen worden. Ze bieden nog steeds de meeste flexibiliteit. Maar native applications bouwen is tijdrovend en duur. 
Voor projecten met een kleiner budget of een scherpe deadline zullen PWA’s de markt domineren.
PWA's bieden vandaag al de snelste ervaring maar er ontbreken nog functionaliteiten. Er wordt verwacht dat dit de komende jaren zal verbeteren.






%------------------------------------------------------------------------------
% Referentielijst
%------------------------------------------------------------------------------
% TODO: de gerefereerde werken moeten in BibTeX-bestand ``voorstel.bib''
% voorkomen. Gebruik JabRef om je bibliografie bij te houden en vergeet niet
% om compatibiliteit met Biber/BibLaTeX aan te zetten (File > Switch to
% BibLaTeX mode)

\phantomsection
\printbibliography[heading=bibintoc]

\end{document}
