%---------- Inleiding ---------------------------------------------------------

\section{Introductie} % The \section*{} command stops section numbering
\label{sec:introductie}

Op een desktop is het gebruikelijk om taken uit te voeren in de browser. Voorbeelden hiervan zijn Gmail en Google drive. Dit is echter nog niet het geval op mobiele toestellen. Elke mobiele service heeft momenteel een native applicatie nodig om een goede omzetting ratio te hebben. Met een native applicatie is het ook makkelijker om gebruikers op een platform te houden.
Dit geen goedkope of snelle oplossing. Er moet gelijkaardige code geschreven worden voor meerdere platformen. Digitale agentschappen en andere software ontwikkeling bedrijven willen vaak een zo snel mogelijke service bieden aan hun klanten, dit is momenteel moeilijk.

Progressive web apps kunnen voor deze problemen een oplossing bieden.
Bij het ontwikkelen van een mobiele applicatie zijn vaak functies, aangeboden door het besturingssysteem, nodig. Voorbeelden hiervan zijn zijn: locatievoorzieningen, offline gebruik, camera, …

In deze thesis zullen volgende onderzoeksvragen onderzocht worden
\begin{itemize}
    \item Hoe groot is de barrière om van een web applicatie  een progressive web app te maken?
    \item Wat zijn de beperkingen van een progressive web app?
    \item Hoe kan een PWA alle functionaliteiten gebruiken die beschikbaar zijn voor native applicaties?
    \item Hoe staan de verschillende besturingssystemen ten opzichte van progressive web apps?
    \item Welke andere technologieën kunnen er gebruikt worden om applicaties te ontwikkelen voor meerdere platformen waarbij er maar 1 codebase is?
\end{itemize}

 



%---------- Stand van zaken ---------------------------------------------------

\section{State-of-the-art}
\label{sec:state-of-the-art}
\subsection{Wat is een PWA}

Een PWA is een website gebouwd met web technologieën die zich gedraagt als een native app maar waarbij er niet door het installatie proces moet gegaan worden zoals bij native applicaties.
\autocite{Sayali2018}

Een PWA is een 'enhanced website', dit is een website met een paar extra bestanden die er voor zorgen dat de site extra functionaliteiten heeft. Volgende bestanden zijn nodig om een van een website een PWA te maken
\begin{itemize}
    \item Manifest.json – Dit is een bestand waar je enkele eigenschappen van de applicatie instelt zoals: app icoon, startpagina, kleurschema, …
    \item Service worker – Dit is een bestand waarbij je zelf caching kan doen. Hierdoor kan, eens een website geladen is geweest, de site offline gebruikt worden.
\end{itemize}
\autocite{Harris2017}

\subsection{Welke functies van een besturingssysteem kan een PWA gebruiken}
Niet alle besturingssystemen geven evenveel functies die gebruikt kunnen worden vanuit een PWA: IOS and Android stellen volgende functies ter beschikking
\begin{itemize}
    \item Notificaties op het toestel
    \item Locatie voorziening
    \item Camera
    \item Gebruik van de sensoren van het toestel
    \item Audio output
    \item Betaling systemen (Android pay bij Android, Apple pay bij IOS)
    \item Spraakinput
    \item Bluetooth (enkel Android)
\end{itemize}
Andere functies waar native apps wel toegang tot hebben zijn niet beschikbaar voor PWA's
\begin{itemize}
    \item Toegang tot contacten
    \item Toegang tot de kalender
    \item Toegang tot alarmen
    \item Toegang tot telefonie data
    \item Berichten
    \item Bel functie
    \item Toegang tot low level hardware sensoren
    \item Camera (video’s)
    \item Maximum storage van 50Mb op IOS
    \item Geen widgets
\end{itemize}

\autocite{Malavolta2016}
\autocite{Destrebecq2018}

\subsection{Waarom progressive web applications}

De verwachtingen die een gebruiker heeft van een website of mobiele service zijn hoger dan ooit. Als je website niet binnen 3 seconden geladen is zal je al 53%
van de gebruikers verliezen. Dit kan voorkomen worden door progressive web apps. Als de website 1 maal geladen is geweest kan de site opgeslagen worden in het cache geheugen. Dit zorgt voor een snellere ervaring.
\autocite{Google2017}

Gebruikers raken gerustreerd als het gedrag van een service anders is op de verschillende platformen. Dit verschil komt er omdat dit totaal verschillende code-bases zijn, die vaak door verschillende teams ontwikkeld en onderhouden worden. Met progressive web applicaties wordt er 1 codebase geschreven die op alle platformen gebruikt wordt. Hierdoor zal het systeem op elk platform heel gelijkaardig werken.
\autocite{Google2019}

Volgens Google zijn er 3 grote redenen om over te schakelen naar PWA’s

\begin{itemize}
    \item Betrouwbaarheid – Door het cache geheugen zal een gebruiker nooit een pagina terechtkomen met een melding dat er geen internet connectie is.
    \item Snelheid – Een PWA kan sneller zijn dan een gewone website door het gebruik van cache storage. Een PWA kan sneller zijn dan een native app doordat het een veel kleiner bestand is.
    \item Aantrekkelijkheid – een PWA kan aanvoelen als een native applicatie.
\end{itemize}
\autocite{GooglePwa2019}

\subsection{Beveiliging}
Een Progressive web application moet altijd een HTTPS verbinding hebben. Dit wil zeggen dat de data die tussen de client en de server verstuurd wordt versleuteld is.
\autocite{Durumeric2013}
Dit is een stap in de juiste richting maar het zorgt er dus niet voor dat elke PWA een veilige applicatie is. PWA’s worden vandaag gebruikt om gebruikers op te lichten. Een vaak gebruikte techniek is dat een bestaande native applicatie wordt nagemaakt. Op deze manier proberen ze wachtwoorden en betalingsdetails te verkrijgen.
\autocite{Lee2018}

\subsection{Native containers}
Niet alle problemen kunnen opgelost worden via het web de dag van vandaag. Voor sommige taken zijn nog steeds native applicaties nodig. Als er een PWA moet gemaakt worden met de functies van een native applicatie, kan deze PWA ontsloten worden door een native wrapper.
Een nadeel hiervan is dat de grootte van een PWA drastisch verhoogt.
Een ander nadeel is dat er twee of meer codebases onderhouden moeten worden, één van de site en minstens één van de wrappers.
\autocite{Richard2019}



% Voor literatuurverwijzingen zijn er twee belangrijke commando's:
% \autocite{KEY} => (Auteur, jaartal) Gebruik dit als de naam van de auteur
%   geen onderdeel is van de zin.
% \textcite{KEY} => Auteur (jaartal)  Gebruik dit als de auteursnaam wel een
%   functie heeft in de zin (bv. ``Uit onderzoek door Doll & Hill (1954) bleek
%   ...'')


%---------- Methodologie ------------------------------------------------------
\section{Methodologie}
\label{sec:methodologie}
Het onderzoek zal uit 3 delen bestaan. 

Een eerste deel waar ik een PWA maak die gebruik maakt van de functionaliteiten van een besturingssysteem die aanspreekbaar zijn vanuit een progressive web app. 

In het tweede luik van het onderzoek zal er gezocht worden naar hoe de functies die normaal enkel toegangkelijk zijn vanuit een native app toch gebruikt kunnen worden binenen een PWA. Voorbeelden hiervan zijn bluetooth, NFC, … 
Er zal een PWA ontwikkeld worden waar deze functionaliteiten gebruikt worden door deze progressive web app te ontsluiten in een Native shell.

Na het voeren van deze twee praktische onderzoeken zullen er ook theoretische onderzoeken gevoerd worden. 
Er zal bekeken worden welke besturingssystemen de meeste functionaliteiten bieden waar progressive web apps gebruik kunnen van maken. 
Er zal ook in kaart gebracht worden wat de zwaktes en limiterende factoren zijn van een PWA ten opzichten van een native applicatie.
In een laatste deel van het onderzoek zal er gezocht worden naar andere technologieën waarbij applicaties kunnen ontwikkeld worden voor meerdere platformen met één codebase. De ervaring die deze technologieën bieden voor de eindgebruiker en de ontwikkelaar zullen bekeken worden.



\subsection{Technologieën}

Progressive web apps kunnen gemaakt worden met de tools die gebruikt worden om gewone webapplicaties te bouwen. 
Voor de onderzoeken zal voor de eerste fase HTML, CSS en JavaScript gebruikt worden. Eventueel kan er ook gebruik gemaakt worden van een javascript library zoals React.
Er zijn verschillende tools beschikbaar voor het creëren van PWA’s éen van deze tools is Ionic.
Voor het maken van een native shell zal er waarschijnlijk gebruik gemaakt worden van react native. Dit is een framework waarbij éen codebase gebruikt wordt om apps te ontwikkelen voor IOS en Android. Hier moet er nog steeds vaak gelijkaardige code geschreven worden om de code te laten werken op de verschillende platformen.




%---------- Verwachte resultaten ----------------------------------------------
\section{Verwachte resultaten}
\label{sec:verwachte_resultaten}


De verwachting is dat het maken van een website die voldoet aan de kenmerken van een PWA niet zo moeilijk zal zijn. Effectief gebruik maken van de extra functionaliteiten die een progressive web app biedt zal minder eenvoudig zijn. De verwachting is dat het proces waarbij er manueel data uit API-verzoeken gecached wordt moeilijk en complex zal zijn.
Het ontwikkelen van een PWA die wel gebruik kan maken van alle features die beschikbaar zijn voor native apps zal ingewikkelder zijn en zal meer tijd vragen. 
Bij het maken van zo een ‘web based hybrid mobile app’  wordt de PWA in een bestaande native container geplaatst. Deze applicatie kan dan in de app-stores geüpload worden.
\autocite{Malavolta2016}
Er wordt verwacht dat er bij de besturingssystemen een duidelijk verschil zal zijn. 
Google, de ontwikkelaar van Android, heeft de revolutie die progressive web applications wel zijn gestart. Ze proberen zoveel mogelijk ontwikkelaars te overtuigen om PWA’s te maken. Hierdoor wordt er verwacht dat Android meer functies van het besturingssysteem open zal stellen voor PWA’s.
Het is daarentegen bekend dat Apple toegang tot het besturingssysteem zoveel mogelijk wil beperken. Ik verwacht een duidelijk verschil tussen deze twee besturingssystemen. 
\autocite{Hansen2017}




%---------- Verwachte conclusies ----------------------------------------------
\section{Verwachte conclusies}
\label{sec:verwachte_conclusies}

Native applicaties zullen waarschijnlijk niet snel vervangen worden. Ze bieden nog steeds de meeste flexibiliteit. Maar native applicaties bouwen is tijdrovend en duur. 
Voor projecten met een kleiner budget of een scherpe deadline zullen PWA’s de markt domineren.
Progressive web apps bieden vandaag al de snelste ervaring maar er ontbreken nog functionaliteiten. Ik verwacht dat dit de komende jaren zal verbeteren.


