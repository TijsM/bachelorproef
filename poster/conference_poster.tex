
\documentclass[a0,portrait]{a0poster}

\usepackage{multicol} % This is so we can have multiple columns of text side-by-side
\columnsep=100pt % This is the amount of white space between the columns in the poster
\columnseprule=3pt % This is the thickness of the black line between the columns in the poster

\usepackage[svgnames]{xcolor} % Specify colors by their 'svgnames', for a full list of all colors available see here: http://www.latextemplates.com/svgnames-colors

\usepackage{times} % Use the times font
%\usepackage{palatino} % Uncomment to use the Palatino font

\usepackage{graphicx} % Required for including images
\graphicspath{{figures/}} % Location of the graphics files
\usepackage{booktabs} % Top and bottom rules for table
\usepackage[font=small,labelfont=bf]{caption} % Required for specifying captions to tables and figures
\usepackage{amsfonts, amsmath, amsthm, amssymb} % For math fonts, symbols and environments
\usepackage{wrapfig} % Allows wrapping text around tables and figures
\usepackage[export]{adjustbox}

\begin{document}

\begin{minipage}[t]{0.75\linewidth}
\VeryHuge \color{HoGentAccent1} \textbf{De interactie van progressive web applications met het besturingssysteem en een proof-of-concept} \color{Black}\\ % Title
%\Huge\textit{Ondertitel (eventueel)}\\[2.4cm] % Subtitle
\huge \textbf{Martens Tijs, Floré Simon, Samyn Karine}\\[0.5cm] % Author(s)
\huge Hogeschool Gent, Valentin Vaerwyckweg 1, 9000 Gent\\[0.4cm] % University/organization
\Large \texttt{tijs.martens@hogent.be} \\
\end{minipage}
%
\begin{minipage}[t]{0.25\linewidth}
\includegraphics[width=13cm,right]{figures/HOGENT_Logo_Pos_rgb.png} 

\end{minipage}

\vspace{1cm}

%----------------------------------------------------------------------------------------

\begin{multicols}{2}
%----------------------------------------------------------------------------------------
%	ABSTRACT
%----------------------------------------------------------------------------------------

\color{HoGentAccent1}

\begin{abstract}
	PWA's zijn webapplicaties die gebruik maken van moderne web-technologieën om een ervaring aan te bieden die gelijkaardig is aan die van een native applicatie. 
	De technologie biedt mogelijkheden om problemen waar ontwikkelaars en digitale agentschappen al jaren mee kampen, op te lossen.  Een PWA is echter nog steeds gelimiteerd op bepaalde vlakken. 
	Het is dus belangrijk dat er in kaart gebracht werd wat wel en niet bereikt kan worden met de technologie.
	%\autocite{googleTrends2020}
\end{abstract}


\color{HoGentAccent1} 
\section*{Introductie}
\color{black}
\color{black}

	In deze scriptie werd onderzocht wat technisch wel en niet mogelijk is met een Progressive Web Application (PWA). Er zal gestart worden met een literatuurstudie, de voor- en nadelen van de technologie zullen hier opgelijst en besproken worden.
	In dit hoofdstuk zal er ook op zoek gegaan worden naar andere technologieën die wel aan de tekortkoming van PWA's kunnen voldoen.
	Vervolgens zal er op een praktische wijze onderzocht worden wat er nodig is om een bestaande traditionele webapplicatie om te vormen tot een PWA die installeerbaar is en offline gebruikt kan worden.
	Ten slotte zal een proof-of-concept uitgewerkt worden waarbij verschillende web-technologieën gebruikt zullen worden om een volwaardige video conference applicatie te ontwikkelen. 

\color{Black}
\color{HoGentAccent1} 
\section*{Experimenten}
\color{black}


\color{HoGentAccent1} 
\section*{Conclusies}
\color{black}

\color{HoGentAccent1} 
\section*{Toekomstig onderzoek}
\color{black}


\end{multicols}
\end{document}