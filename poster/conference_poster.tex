
\documentclass[a0,portrait]{a0poster}

\usepackage{multicol} % This is so we can have multiple columns of text side-by-side
\columnsep=100pt % This is the amount of white space between the columns in the poster
\columnseprule=3pt % This is the thickness of the black line between the columns in the poster

\usepackage[svgnames]{xcolor} % Specify colors by their 'svgnames', for a full list of all colors available see here: http://www.latextemplates.com/svgnames-colors

\usepackage{times} % Use the times font
%\usepackage{palatino} % Uncomment to use the Palatino font

\usepackage{graphicx} % Required for including images
\graphicspath{{figures/}} % Location of the graphics files
\usepackage{booktabs} % Top and bottom rules for table
\usepackage[font=small,labelfont=bf]{caption} % Required for specifying captions to tables and figures
\usepackage{amsfonts, amsmath, amsthm, amssymb} % For math fonts, symbols and environments
\usepackage{wrapfig} % Allows wrapping text around tables and figures
\usepackage[export]{adjustbox}

\usepackage{xcolor}
\usepackage{colortbl}

\begin{document}

\begin{minipage}[t]{0.75\linewidth}
\VeryHuge \color{HoGentAccent1} \textbf{De interactie van progressive web applications met het besturingssysteem en een proof-of-concept} \color{Black}\\ % Title
%\Huge\textit{Ondertitel (eventueel)}\\[2.4cm] % Subtitle
\huge \textbf{Martens Tijs, Floré Simon, Samyn Karine}\\[0.5cm] % Author(s)
\huge Hogeschool Gent, Valentin Vaerwyckweg 1, 9000 Gent\\[0.4cm] % University/organization
\Large \texttt{tijs.martens@student.hogent.be} \\
\end{minipage}
%
\begin{minipage}[t]{0.25\linewidth}
\includegraphics[width=13cm,right]{figures/HOGENT_Logo_Pos_rgb.png} 

\end{minipage}

\vspace{1cm}

%----------------------------------------------------------------------------------------

\begin{multicols}{2}
%----------------------------------------------------------------------------------------
%	ABSTRACT
%----------------------------------------------------------------------------------------

\color{HoGentAccent1}

\begin{abstract}
	PWA's zijn webapplicaties die gebruik maken van moderne web-technologieën om een ervaring aan te bieden die gelijkaardig is aan die van een native applicatie. 
	De technologie biedt mogelijkheden om problemen waar ontwikkelaars en digitale agentschappen al jaren mee kampen, op te lossen.  Een PWA is echter nog steeds gelimiteerd op bepaalde vlakken. 
	Het is dus belangrijk dat er in kaart gebracht werd wat wel en niet bereikt kan worden met de technologie.
\end{abstract}


\color{HoGentAccent1} 
\section*{Introductie}
\color{black}
\color{black}

	In deze scriptie werd onderzocht wat technisch wel en niet mogelijk is met een Progressive Web Application (PWA). Er werd gestart met een literatuurstudie, de voor- en nadelen van de technologie zullen hier opgelijst en besproken worden.
	In dit hoofdstuk werd op zoek gegaan naar andere technologieën die wel aan de tekortkoming van PWA's kunnen voldoen.
	Vervolgens werd er op een praktische wijze onderzocht wat er nodig is om een bestaande traditionele webapplicatie om te vormen tot een PWA die installeerbaar is en offline gebruikt kan worden.
	Ten slotte werd een proof-of-concept uitgewerkt waarbij verschillende web-technologieën gebruikt zullen worden om een volwaardige video conference applicatie te ontwikkelen. 

\color{Black}
\color{HoGentAccent1} 
\section*{Wat is een PWA}
\color{black}

	Een PWA is een webapplicatie die gebruik maakt van moderne web API’s om functies aan te bieden die voordien enkel beschikbaar waren voor native applicaties. PWA’s combineren de sterktes van het web en de sterktes van native applicaties
	
	\subsection{Service workers}
		De service worker is een script dat veel functionaliteiten beschikbaar maakt die voordien enkel beschikbaar waren voor native applicaties. Een service worker is een web worker die tussen het netwerk en de applicatie wordt geplaatst. Dit zorgt ervoor dat de service worker inkomende en uitgaande netwerkverzoeken kan controleren en eventueel manipuleren.
		
		Een service worker zorgt ervoor dat een PWA van volgende functionaliteiten kan genieten:
		\begin{itemize}
		   	\item Offline gebruik
		   	\item Caching
		   	\item Push-notificaties
		   	\item Add-to-homescreen (A2HS)
		\end{itemize}

\color{Black}
\color{HoGentAccent1} 
\section*{Functionaliteiten en besturingssystemen}
\color{black}

	\begin{tabular}{lllllll}
		Real time communication &	Edge & Firefox & Chrome & Safari & Android (Chrome) & iOS (Safari) \\
		&Ja   & Ja      & Ja     & Ja     & Ja              & Ja \\       
		Push notificaties&	Edge & Firefox & Chrome & Safari & Android (Chrome) & iOS (Safari) \\
		&Ja   & Ja      & Ja     & Nee     & Ja              & Nee     \\
		Offline beschikbaar &	Edge & Firefox & Chrome & Safari & Android (Chrome) & iOS (Safari) \\
		&Ja   & Ja      & Ja     & Ja     & Ja              & Ja     \\
		In-app betalingen &	Edge & Firefox & Chrome & Safari & Android (Chrome) & iOS (Safari) \\
		&Ja   & Nee      & Ja     & Ja     & Ja              & Ja     \\
	\end{tabular}	
	
	Het is opvallend hoeveel functies beschikbaar zijn voor het web. Slechts een beperkt aantal functionaliteiten die beschikbaar zijn voor native applicaties heeft geen variant voor webapplicaties.

	\subsection{Web}
		We kunnen concluderen dat de browsers die Google maakt (Google Chrome, Google Chrome for Android) een betere ondersteuning geeft dan de browsers die Apple maakt (Safari, Safari iOS). Deze trend is zowel te zien bij de mobiele browsers als de browsers voor desktop.
	
	\subsection{Mobiel}
		Er kan geconcludeerd worden dat Android meer open staat voor intergraties met PWA’s dan iOS. In deze trend lijkt er niet direct verandering te komen. Google, de ontwikkelaar van Android, is vaak de eerste om nieuwe web API’s te ondersteunen.
		Langs de andere kant is iOS vaak het laatste besturingssyteem die ondersteuning zal aan bieden voor een web API.

	\subsection{Desktop}
		Er kan geconcludeerd worden dat Windows en macOS een volledig andere filosofie hebben als het aankomt op PWA’s.
		Windows probeert een zo goed mogelijke ondersteuning te geven aan PWA’s. PWA’s worden binnen Windows behandeld als een volwaardig programma. Als een PWA geïnstalleerd is, is het moeilijk te onderscheiden van een ander native programma. PWA’s zijn binnen Windows ook eenvoudig om te installeren. Micorosoft probeert ontwikkelaars ook te ondersteunen in het ontwikkelen van PWA’s. Dit doen ze door een uitgebreide documentatie en tools aan te bieden.
		Apple daarentegen is meer gesloten voor PWA’s. Vanuit de standaard browser kunnen er helemaal geen PWA’s geïnstalleerd worden. Via Google Chrome is dit wel mogelijk.

\color{Black}
\color{HoGentAccent1} 
\section*{Voor- en Nadelen}
\color{black} 
		\begin{tabular}{llll}
			                         			  & Webapplicatie 	 				 & PWA								 & Native applicatie \\
				Bereik                   		   & \cellcolor{green!40}      		 & \cellcolor{green!40}			& \cellcolor{red!50}\\
				Platformonafhankelijkheid   & \cellcolor{green!40}      	  & \cellcolor{orange!50}		& \cellcolor{red!50}\\
				Omzet   						  & \cellcolor{red!50}      		 & \cellcolor{orange!50}		& \cellcolor{green!40}\\
				Bundle size						  & niet van toepassing      		& \cellcolor{green!40}		   & \cellcolor{red!50}\\
				Offline gebruik					 & \cellcolor{red!50}      		    & \cellcolor{green!40}			& \cellcolor{green!40}\\
				Betrokkenheid 					& \cellcolor{red!50}      		   & \cellcolor{orange!50}		 & \cellcolor{green!40}\\
				Kost 								& niet van toepassing     		  & \cellcolor{green!40}		 & \cellcolor{red!50}\\
				Deployment 						& \cellcolor{green!40}      	  & \cellcolor{green!40}		 & \cellcolor{red!50}\\
				Updates	   						& \cellcolor{green!40}      	  & \cellcolor{green!40}		 & \cellcolor{red!50}\\
				Cameragebruik                  & \cellcolor{red!50}      		 & \cellcolor{red!50}			& \cellcolor{green!40}\\
				Controle over platformen   	& \cellcolor{red!50}      	  & \cellcolor{red!50}		& \cellcolor{green!40}\\
				Functies besturingssysteem  & \cellcolor{red!50}      		 & \cellcolor{orange!50}		& \cellcolor{green!40}\\
				Browserondersteuning		 & \cellcolor{green!40}     		& \cellcolor{orange!50}		   &niet van toepassing\\
				Aanwezigheid in app-stores	 & \cellcolor{red!50}      		    & \cellcolor{red!50}			& \cellcolor{green!40}\\
		\end{tabular}

\color{Black}
\color{HoGentAccent1} 
\section*{A2HS}
\color{black}
	
	Onderzoek toont aan dat een applicatie die werd toegevoegd aan het startscherm vaker gebruikt zal worden en dat de sessies langer zullen zijn. Als uitgever van een PWA heb je er dus alle belang bij dat zoveel mogelijk gebruiker de PWA toevoegen aan hun startscherm. Uit de proof-of-concepts in deze thesis blijkt dat meer gebruiker de applicatie zullen toevoegen aan hun startscherm als hier zelf kunnen voor kiezen. Minder gebruikers voegden de applicatie toe als er prominente pop-up's waren met de vraag de om de PWA toe te voegen.

\color{HoGentAccent1} 
\section*{Conclusies}
\color{black}

	Heel wat toepassingen kunnen ontwikkeld worden aan de hand van een PWA. Bepaalde applicaties zijn echter meer geschikt om geïmplementeerd te worden als PWA dan andere.
	
	In deze thesis werd beschreven in welke situaties een PWA voordelig in ten opzichte van andere technologieën. Ook werd er een overzicht gemaakt van de functionaliteiten die beschikbaar zijn voor PWA's op de verschillende platformen.
	
	Door deze secties te bekijken kan er voor elk project bepaald worden als een PWA de juiste benadering is voor een bepaalde toepassing. 
	
	PWA's hebben enkele unieke voordelen ten opzichte van native applicaties en traditionele webapplicaties. 

	Als er voor een project rond de limitaties, die er wel zijn, heen gewerkt kan worden, kan er met een PWA een uniek ervaring ontwikkeld worden.
	Deze ervaring zal snel werken op alle mogelijke toestellen en zal van verschillende voordelen kunnen genieten ten opzichte van traditionele webapplicaties en native applicaties.
	
	


\color{HoGentAccent1} 
\section*{Toekomstig onderzoek}
\color{black}

	Veel technologieën die gebruikt kunnen worden door PWA’s zijn relatief nieuw en zijn nog niet grondig onderzocht. In deze thesis werd de nadruk gelegd op welke functies er wel en niet kunnen gebruikt worden door PWA’s. Om hier een goed beeld van te schetsen dient er nog onderzoek gedaan te worden naar de performantie van deze web-API’s.
	In de proof-of-concept werd webRTC gebruikt om een peer-to-peer connectie op te zetten. Dit is een technologie met heel veel mogelijkheden. Er kan onderzoek gedaan worden naar wat er bereikt kan worden met webRTC en hoe performant dit is.
	In deze thesis werd verschillende keren aangehaald dat PWA’s push notificaties kunnen gebruiken. Er zou onderzoek gedaan kunnen worden naar hoe deze het best ingezet worden om de engagement en de conversion van een applicatie te verhogen.


\end{multicols}
\end{document}