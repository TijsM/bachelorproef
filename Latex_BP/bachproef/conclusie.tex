%%=============================================================================
%% Conclusie
%%=============================================================================

\chapter{Conclusie}
\label{ch:conclusie}

% Trek een duidelijke conclusie, in de vorm van een antwoord op de
% onderzoeksvra(a)g(en). Wat was jouw bijdrage aan het onderzoeksdomein en
% hoe biedt dit meerwaarde aan het vakgebied/doelgroep? 
% Reflecteer kritisch over het resultaat. In Engelse teksten wordt deze sectie
% ``Discussion'' genoemd. Had je deze uitkomst verwacht? Zijn er zaken die nog
% niet duidelijk zijn?
% Heeft het onderzoek geleid tot nieuwe vragen die uitnodigen tot verder 
%onderzoek?
	  

\section{Resultaten}
%todo: antwoord formuleren op hoofd OV "welke applicaties kunnen er ontwikkeld worden als PWA
%vergelijking met verwacht resultaten

	\subsection{Welke applicaties kunnen er ontwikkeld worden als PWA}
		
		In sectie ~\ref{ch:pwaAlternatieven} werd beschreven in welke situaties het gebruik van PWA's het meest voordelig is en in sectie ~\ref{ch: BesturingssystemenEnPWAs} werd er een overzicht gemaakt van welke functionaliteit met een PWA geïmplementeerd kan worden en welke niet.
		
		Door deze twee onderzoeken samen te leggen kan er geconcludeerd worden welke use-cases optimaal zijn voor PWA's.
		
		
		%todo: VEEEL uitgebreider
		% veel apps kunnen als pwa geaakt worden en zullen wel wat voordeel hebben
		%in bepaalde use cases (e commerce) heeft dit heeeeeeel grote voordelen.
		
				
		
		
	\subsection{Functionaliteit van PWA's en besturingssystemen}
	
		In hoofdstuk 2 werd er aangetoond dat PWA's kunnen genieten van veel verschillende web-API's om native functionaliteiten te implementeren. Applicaties die afhankelijk zijn van de camera of van locatievoorzieningen kunnen perfect geïmplementeerd worden als PWA. Ook meer geavanceerde functionaliteiten zoals in app betalingen en virtual reality kunnen in theorie geïmplementeerd worden.
		
		De ondersteuning van deze krachtige web-API's is helemaal niet consistent, wat het moeilijk maakt om een PWA te ontwikkelen die afhankelijk is van verschillende web-API's. 
		
		In het algemeen kan er gesteld worden dat Google en Microsoft een goede ondersteuning bieden voor PWA's. Ze ondersteunen niet enkel PWA's maar beide bedrijven werken actief aan het uitbreiden van de functionaliteiten die beschikbaar zijn voor het web. 
		Apple daarentegen is conservatiever op vlak van het ondersteunen van PWA's. Het bedrijf is trager in het ondersteunen van nieuwe web-API's. 
		Het duurde lang voor IOS service workers ondersteunde, dit gebeurde pas in 2020. Ook worden bepaalde belangrijke web-API's niet ondersteund, het grootste voorbeeld hiervan is de push API.
					
	\subsection{Beperkingen van een PWA}
		content
	
	\subsection{PWA alternatieven}
		
		Er zijn verschillende technologieën en frameworks waarbij er met één codebase een applicatie ontwikkeld kan worden voor verschillende platformen. 
		
		Er zijn verschillende parameters die bepalen welke benadering te juiste is voor een bepaald project. 
	
		Een PWA is een goede oplossing voor applicaties waarbij de go to market time zo laag mogelijk moet zijn en er snel en eenvoudig updates moeten uitgevoerd worden. Een groot voordeel van PWA's is ook dat het gevonden kan worden in zoekmachines.
		
		Een hybride oplossing is interessant als er zowel een webapplicatie als een native applicatie nodig is. Hier worden zowel prestaties en user experience opgeofferd, maar de applicatie is wel op alle platformen beschikbaar. 
		
		Als er een native applicatie verwacht wordt zijn er verschillende benaderingen om dit te implementeren. Hier zal de achtergrond en voorkennis van de ontwikkelaar een grote impact hebben op wat het juiste framework is.
	
	\subsection{Welke stappen zijn er nodig om een traditionele website om te vormen tot een PWA}
		
		Bij deze onderzoeksvraag werd een PWA gezien als een webapplicatie die kan geïnstalleerd worden op een toestel en die offline gebruikt kan worden.
		
		Dit is kan eenvoudig bereikt worden met aan de hand van moderne frameworks zoals React. De service worker wordt automatisch gegenereerd. Ook het app-manifest kan gegenereerd worden. Er zijn verschillende platformen die https-hosting aanbieden. 
		
		De drie criteria om een PWA te installeren kunnen dus eenvoudig bereikt worden.
	
\section{Niet-kwantificeerbare resultaten}
	In de proof-of-concept werd aangetoond dat geavanceerde concepten zoals peer-to-peer verbindingen en real-time communicatie mogelijk zijn voor PWA's. Een PWA kan dus veel functionaliteiten implementeren die voordien enkel beschikbaar waren voor native applicaties.
	
	Uit de proof-of-concept bleek ook dat een PWA geen 'native-gevoeld' kan bieden. 
	Dit heeft verschillende oorzaken, zo heeft de gebruiker op een bepaald platform een verwachtingspatroon van hoe een applicatie zich zal gedragen op dat platform. Dit is voor elk platform anders. Bij PWA's wordt er 1 user interface gemaakt voor alle platformen.
	
	Een andere oorzaak is dat het niet eenvoudig is om een consistente user interface te ontwikkelen die native aanvoelt voor alle mogelijke schermgroottes.
	
\section{Vergelijking met verwacht resultaten}
	Er werd verwacht dat het ontwikkelen van een PWA met service worker functionaliteit een complexe opdracht zou zijn.
	Deze verwachting klopte niet, de meest voorkomend functionaliteiten is er een duidelijk documentatie. Ook maakt workbox het heel eenvoudig om caching toe te passen in een applicatie.
	

\section{Verder onderzoek}
	Veel technologieën die gebruikt kunnen worden door PWA's zijn relatief nieuw en zijn nog niet grondig onderzocht.
	In deze thesis werd de nadruk gelegd op welke functies er wel en niet kunnen gebruikt worden door PWA's. Om hier een goed beeld van de schetsen kan er nog onderzoek gedaan worden naar de performantie van deze web-API's.
	
	In de proof of concept in deze scriptie werd webRTC gebruikt om een peer-to-peer connectie op te zetten. Dit is een technologie met heel veel mogelijkheden, er kan onderzoek gedaan worden naar wat er bereikt kant worden met webRTC en hoe performant dit is.