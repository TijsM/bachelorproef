\chapter{Conclusie}
\label{ch:conclusie}

% Trek een duidelijke conclusie, in de vorm van een antwoord op de
% onderzoeksvra(a)g(en). Wat was jouw bijdrage aan het onderzoeksdomein en
% hoe biedt dit meerwaarde aan het vakgebied/doelgroep? 
% Reflecteer kritisch over het resultaat. In Engelse teksten wordt deze sectie
% ``Discussion'' genoemd. Had je deze uitkomst verwacht? Zijn er zaken die nog
% niet duidelijk zijn?
% Heeft het onderzoek geleid tot nieuwe vragen die uitnodigen tot verder 
%onderzoek?
	  

\section{Resultaten}
%todo: antwoord formuleren op hoofd OV "welke applicaties kunnen er ontwikkeld worden als PWA
%vergelijking met verwacht resultaten

	\subsection{Welke applicaties kunnen er ontwikkeld worden als PWA}
	
		Heel wat applicaties kunnen ontwikkeld worden aan de hand van een PWA. Echter zijn bepaalde applicaties meer geschikt om geïmplementeerd te worden als PWA dan andere.
		
		In sectie ~\ref{ch:pwaAlternatieven} werd beschreven in welke situaties het gebruik van PWA's het meest voordelig is en in sectie ~\ref{ch: BesturingssystemenEnPWAs} werd er een overzicht gemaakt van welke functionaliteit met een PWA geïmplementeerd kan worden en welke niet.
		
		Door deze twee onderzoeken samen te leggen kan er geconcludeerd worden welke use-cases optimaal zijn voor PWA's.
		
		Volgende toepassingen kunnen optimaal gebruik maken van de functionaliteiten die PWA's bieden.
		
		Een van de grote voordelen van een PWA is dat hij gevonden kan worden aan de hand van een zoekmachine. Dit is een groot voordeel voor E-commerce platformen en social media.
		Applicaties die hier minder voordeel bij hebben zijn bijvoorbeeld 
		
		Ook voor toepassingen waarbij de time to market zo laag mogelijk moet zijn, bieden PWA's voordelen. Het voorbeeld van de 'the coronavirus app' aangehaald in sectie ~\ref{ch: Voorbeelden} is hier een goed voorbeeld van. Het succes van de PWA is grotendeels te danken aan de snelheid waarmee de app online stond.

	\subsection{Functionaliteit van PWA's en besturingssystemen}
	
		In sectie ~\ref{ch: BesturingssystemenEnPWAs} werd er aangetoond dat PWA's kunnen genieten van veel verschillende web-API's om native functionaliteiten te implementeren. Applicaties die afhankelijk zijn van de camera of van locatievoorzieningen kunnen perfect geïmplementeerd worden als PWA. Ook meer geavanceerde functionaliteiten zoals in app betalingen en virtual reality kunnen geïmplementeerd worden.
		
		De ondersteuning van deze krachtige web-API's is echter niet consistent Dit maakt het moeilijk om een PWA te ontwikkelen die afhankelijk is van verschillende web-API's. 
		
		In het algemeen kan er gesteld worden dat Google en Microsoft een goede ondersteuning bieden voor PWA's. Ze ondersteunen niet enkel veel functionaliteiten voor PWA's maar beide bedrijven werken actief aan het uitbreiden van de functionaliteiten die beschikbaar zijn voor het web. 
		
		Apple daarentegen is conservatiever op vlak van het ondersteunen van PWA's. Het bedrijf is trager in het ondersteunen van nieuwe web-API's. 
		Het duurde lang voor IOS service workers ondersteunde, dit gebeurde pas in september 2019. \autocite{Apple2020} Ook worden bepaalde belangrijke web-API's niet ondersteund, het grootste voorbeeld hiervan is de push API.
					
	\subsection{Beperkingen van een PWA}
		In sectie ~\ref{ch: beperkingenPWA} werden de beperkingen van PWA's uitgebreid toegelicht. 
	
		De grootste beperking die PWA's vandaag kent is de zwakke ondersteuning op IOS toestellen.
		
		In deze scriptie werd aangetoond dat Android gebruikers geen of amper verschil merken tussen een PWA en een native applicatie. Android  
		
		Als IOS gebruikers dezelfde PWA te zien kregen vonden ze dat dit niet vergelijkbaar was aan een native applicatie. Dit komt omdat IOS veel webAPI's niet ondersteund.
		
	 
	
	\subsection{PWA alternatieven}
		
		Er zijn verschillende technologieën en frameworks waarbij er met één codebase een applicatie ontwikkeld kan worden voor verschillende platformen. 
		
		Er zijn verschillende parameters die bepalen welke benadering de juiste is voor een bepaald project. 
	
		Een PWA is een goede oplossing voor applicaties waarbij de go to market time zo laag mogelijk moet zijn en er snel en eenvoudig updates moeten uitgevoerd worden. Een groot voordeel van PWA's is ook dat het gevonden kan worden in zoekmachines.
		
		Een hybride oplossing is interessant als er zowel een webapplicatie als een native applicatie nodig is. Hier worden zowel prestaties en user experience opgeofferd, maar de applicatie is wel op alle platformen beschikbaar. 
		
		Als het belangrijk is om applicatie in de app stores te hebben zijn er verschillende oplossingen om een applicatie te maken die toch op verschillende platformen kan werken. Hier zal de achtergrond en voorkennis van de ontwikkelaar een grote impact hebben op wat het juiste framework is.
		
		Een uitgebreid overzicht van de welke benadering de juiste is voor een project kan gevonden worden in sectie ~\ref{ch: pwaAlternatieven}.
	
	\subsection{Welke stappen zijn er nodig om een traditionele website om te vormen tot een PWA}
		
		Bij deze onderzoeksvraag werd een PWA gezien als een webapplicatie die kan geïnstalleerd worden op een toestel en die offline gebruikt kan worden.
		
		Dit is kan eenvoudig bereikt worden met aan de hand van moderne frameworks zoals React. De service worker wordt automatisch gegenereerd. Ook het app-manifest kan gegenereerd worden. Er zijn verschillende platformen die https-hosting aanbieden. 
		
		De drie criteria om een PWA te installeren kunnen dus eenvoudig bereikt worden.
	
\section{Vergelijking met verwachte resultaten}
	Er werd verwacht dat het ontwikkelen van een PWA met service worker functionaliteit een complexe opdracht zou zijn.
	Deze verwachting klopte niet, de meest voorkomend functionaliteiten is er een duidelijk documentatie. Ook maakt workbox het heel eenvoudig om caching toe te passen in een applicatie.
	
	Er werd verwacht dat IOS minder ver zou staan op vlak van PWA's. Zowel de literatuurstudie als de user-testen bevestigen deze verwacting.
	

\section{Verder onderzoek}
	Veel technologieën die gebruikt kunnen worden door PWA's zijn relatief nieuw en zijn nog niet grondig onderzocht.
	In deze thesis werd de nadruk gelegd op welke functies er wel en niet kunnen gebruikt worden door PWA's. Om hier een goed beeld van de schetsen kan er nog onderzoek gedaan worden naar de performantie van deze web-API's.
	
	In de proof-of-concept werd webRTC gebruikt om een peer-to-peer connectie op te zetten. Dit is een technologie met heel veel mogelijkheden, er kan onderzoek gedaan worden naar wat er bereikt kant worden met webRTC en hoe performant dit is.
	
	In deze thesis werd verschillende keren aangehaald dat PWA's push notificaties kan gebruiken. Er zou onderzoek gedaan kunnen worden naar hoe deze het best ingezet worden om de engagement en de conversion van een applicatie te verhogen.