%%=============================================================================
%% Voorwoord
%%=============================================================================

\chapter*{\IfLanguageName{dutch}{Woord vooraf}{Preface}}
\label{ch:voorwoord}

%% TODO:
%% Het voorwoord is het enige deel van de bachelorproef waar je vanuit je
%% eigen standpunt (``ik-vorm'') mag schrijven. Je kan hier bv. motiveren
%% waarom jij het onderwerp wil bespreken.
%% Vergeet ook niet te bedanken wie je geholpen/gesteund/... heeft

Deze proef werd geschreven in het kader van het behalen van het diploma 'bachelor in de toegepaste informatica' aan HoGent.

Het onderwerp van deze proef, PWA's, werd voorgesteld door Bothrs. Ik vond dit een heel interessante case en het was dan ook heel boeiend om te ontdekken wat er wel en niet mogelijk is met PWA's.

Ik heb een grote interesse in nieuwe technologieën maar ook in bedrijfsstrategieën en ondernemen. Het onderzoeken van een nieuwe technologie waarbij de time to market drastisch verlaagd zou kunnen worden, was dan ook een perfecte match voor mij. 

Ik zou graag Bothrs bedanken. Dit is de studio die me het onderwerp heeft voorgesteld. 
Tijdens mijn bachelorproef kon ik bij verschillende personen terecht voor hulp of advies. Ook de oprechte interesse van deze personen in de technologie was een grote motivatie tijdens het schrijven van deze scriptie.

Tenslotte wil ik ook graag mijn promotor Karine Samyn bedanken. Ik kreeg op heel regelmatige basis uitgebreide, gerichte en doordachte feedback. Tevens merkte ik bij mevrouw Samyn een oprechte interesse in het onderwerp. Dit alles heeft zeker een positieve impact gehad op het eindresultaat.