%%=============================================================================
%% Samenvatting
%%=============================================================================

% De "abstract" of samenvatting is een kernachtige (~ 1 blz. voor een
% thesis) synthese van het document.

% Deze aspecten moeten zeker aan bod komen:

%✅ - Context: waarom is dit werk belangrijk?
%✅- Nood: waarom moest dit onderzocht worden?
%✅- Taak: wat heb je precies gedaan?
%✅ - Object: wat staat in dit document geschreven?
%✅ - Resultaat: wat was het resultaat?
%✅ - Conclusie: wat is/zijn de belangrijkste conclusie(s)?
%✅- Perspectief: blijven er nog vragen open die in de toekomst nog kunnen  onderzocht worden? Wat is een mogelijk vervolg voor jouw onderzoek?



\chapter*{Samenvatting}

PWA's zijn webapplicaties die gebruik maken van moderne web-technologieën om een ervaring aan te bieden die gelijkaardig is aan die van een native applicatie. Er is steeds meer interesse in PWA's. \autocite{googleTrends2020}


De technologie biedt mogelijkheden om problemen waar ontwikkelaars en digitale agentschappen al jaren mee kampen, op te lossen.  Een PWA is echter nog steeds gelimiteerd op bepaalde vlakken. 
Het is dus belangrijk dat er in kaart gebracht werd wat wel en niet bereikt kan worden met de technologie.

In deze scriptie zal er onderzocht worden wat technisch wel en niet mogelijk is met een PWA. Er zal gestart worden met een literatuurstudie, de voor- en nadelen van de technologie zullen hier opgelijst en besproken worden.
In dit hoofdstuk zal er ook op zoek gegaan worden naar andere technologieën die wel aan de tekortkoming van PWA's kunnen voldoen.
Vervolgens zal er op een praktische wijze onderzocht worden wat er nodig is om een bestaande traditionele webapplicatie om te vormen tot een PWA die installeerbaar is en offline gebruikt kan worden.
Ten slotte zal een proof-of-concept uitgewerkt worden waarbij verschillende web-technologieën gebruikt zullen worden om een volwaardige video conference applicatie te ontwikkelen. 

Een webapplicatie kan eenvoudig omgevormd worden tot een installeerbare, offline ervaring. De proof-of-concept toonde dat aan de hand van web-API's functionaliteit verkregen kan worden die voordien enkel beschikbaar was voor native applicaties.
De ondersteuning van PWA's is op bepaalde platformen echter nog onvoldoende om deze reeds als een vervanger voor native applicaties te zien.

PWA's zijn een relatief nieuwe technologie. Er zijn verschillende web-API's die nog verder onderzocht zouden kunnen worden. Zo zou er verder onderzoek gedaan kunnen worden naar de impact van push notificaties op vlak van user engagement van een webapplicatie en naar de prestaties van bepaalde web-technologieën zoals webRTC.