\section{ntsluiten van een PWA}
In bepaalde situaties zal een PWA niet alle functionaliteiten kunnen ondersteunen om alle requirements van een project te voldoen. Een intuïtieve beslissing zou hier zijn om over te schakelen naar native applicaties en PWA’s links te laten liggen. Dit zou wel als gevolg hebben dat er veel potentiële gebruikers verloren gaan omdat de applicatie dan enkel te vinden is in de app stores en niet via zoekmachines.

In deze sectie van het onderzoek zal er gekeken worden als er een hybride oplossing mogelijk is. Hierbij worden zoveel mogelijk functionaliteiten van de applicatie geïmplementeerd als PWA. Vervolgens kan er een native applicatie gemaakt worden die de ontbrekende functionaliteiten implementeert. Deze native applicatie kan de functionaliteiten die al ontwikkeld zijn in de PWA implementeren als een webview. 

Hier wordt het grote voordeel van deze benadering duidelijk. Gebruikers kunnen de applicatie ontdekken en uittesten via de PWA. Als ze applicatie interessant vinden, kan een volledigere versie gedownload worden via de app-store.

Apple maakt echter wel duidelijk dat het niet de bedoeling is om een website te maken en deze zonder toegevoegde native functionaliteiten in de app-store te plaatsen. De regels van de appstore stellen het volgende:

\textit{“Your app should include features, content, and UI that elevate it beyond a repackaged website. If your app is not particularly useful, unique, or “app-like,” it doesn’t belong on the App Store.”}
\autocite{Apple2020c}


De Google play store heeft hier geen specifieke regels over.
\autocite{GooglePlay2020a}