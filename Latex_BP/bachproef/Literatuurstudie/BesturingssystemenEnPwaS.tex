

\section{Besturingssystemen en PWA’s}

Om te weten te komen voor welke toepassingen een PWA gemaakt kan worden en voor welke toepassingen nog steeds een native applicatie nodig is, is het belangrijk om te weten wat de technische mogelijkheden zijn van een PWA. In deze sectie van de literatuurstudie wordt er bekeken welke functies, die beschikbaar zijn voor native applicaties, al dan niet gebruikt kunnen worden door PWA’s.

Dit onderzoek werd gevoerd met behulp van de website whatwebcando.today en caniuse.com. 

whatwebcando.today is een website die kleine voorbeelden van verschillende technologieën demonstreert. Door deze voorbeelden te testen op verschillende platformen kan er uitgemaakt worden welke technologieën er beschikbaar zijn voor het web, en op welke platformen deze beschikbaar zijn.

Caniuse.com is een website die voor verschillende web-technologieën een overzicht biedt op welke browsers deze technologie gebruikt kan worden en op welke niet. Deze website werd gebruikt om de ondervindingen van de testen die werden uitgevoerd te valideren. 

Een web-API is een API die wordt aangeboden door de browser. Het verschil met web-API’s en traditionele API’s is dat web-API’s lokaal worden aangeboden door de browser en er dus geen internetverbinding nodig is om van deze functionaliteiten te genieten.

\autocite{Mozilla2019c}

Als er meer informatie nodig was over de web API’s werd deze gevonden op developers.Google.com of op developer.mozilla.org. 

Er werd gekeken op welke platformen bepaalde functies wel en niet werkten. De volgende platformen werden onderzocht:
\begin{itemize}
   \item Desktop:
   \begin{itemize}
     \item	Microsoft edge versie 80 op Windows 10
     \item	Mozilla firefox versie 73.0 op Windows 10
     \item	Google Chrome versie 79.0 op Windows 10
     \item  Safari (desktop) versie 13.0.5 op een Macbook Pro met macOS Mojave (10.14.6)
   \end{itemize}
\end{itemize}


\begin{itemize}
   \item Mobiel:
   \begin{itemize}
     \item Google Chrome versie 80 op Android 10 op een OnePlus 6
     \item Safari (mobiel) versie 13 op IOS 13 op een IPhone SE 
     
   \end{itemize}
\end{itemize}

De testen werden uitgevoerd op 7 maart 2020.

De website whatwebcando.today geeft een overzicht van de functionaliteiten aan de hand van een bepaalde structuur. Deze structuur werd overgenomen en ziet er als volgt uit:
   \begin{itemize}
     \item	Media
     \item	Verbinding
     \item	Toestel kenmerken
     \item	Native gedrag
     \item	Besturingssysteem
     \item	Input
     \item	User experience
     \item	Locatie en positionering
     \item	Scherm en output
   \end{itemize}

\subsection{Onderzoek}

\subsubsection{Media}

\paragraph{Video met audio }

Sommige van de meest populaire mobiele applicaties zijn sterk afhankelijk van camera-functionaliteit. Voorbeelden hiervan zijn Snapchat, Instagram, Messenger, WhatsApp, … 

Bij deze applicaties is het belangrijk dat de camera aan volgende vereisten voldoet:

   \begin{itemize}
     \item	Snel en eenvoudig te gebruiken
     \item	Er moet van camera gewisseld kunnen worden
     \item	Er moet ingezoomd kunnen worden
     \item	De flashlight moet gebruikt kunnen worden
   \end{itemize}

De media capture API \autocite{DzungDTran}  maakt het mogelijk om een video die opgenomen wordt met de camera van het toestel te tonen op de webpagina. Deze video kan dan opgeslagen worden in de code en verzonden worden naar een server. 

\autocite{Fransson2017}

De media capture API is ook in staat om aan het toestel te vragen welke camera’s er beschikbaar zijn en dan te verwisselen van camera.

\autocite{Scales2020a}

Ook meer geavanceerde functionaliteiten zijn beschikbaar. Het gedrag van de zoom en de flashlight kan ook programmatisch bepaald worden.

\autocite{Oberhofer2017} \autocite{Ogundipe2018}


Al de belangrijkste functionaliteiten die een gebruiker verwacht, zijn allemaal aanwezig. Applicaties die afhankelijk zijn van video-opnames kunnen dus geïmplementeerd worden als PWA.

\begin{table}[]
	\begin{tabular}{llllll}
		Edge & Firefox & Chrome & Safari & Android (Chrome) & IOS (Safari) \\
		Ja   & Ja      & Ja     & Ja     & Ja               & Ja          
	\end{tabular}	
	\caption{ondersteuning van de Media Capture API}
	\label{ondersteuning van de Media Capture API}
\end{table}



\paragraph{Foto's }

Het vastleggen van foto’s is ook voor veel populaire applicaties belangrijk. Ook dit is een belangrijke functionaliteit voor sociale media applicaties.

Foto’s die gedeeld worden op sociale media moeten vaak van een zo hoog mogelijke kwaliteit zijn. Op native applicaties wordt deze kwaliteit bereikt door volgende eigenschappen: 

 \begin{itemize}
     \item Manuele en automatische focus
     \item Aanpassen van sluitersnelheid
     \item Aanpassen van witbalans
     \item Aanpassen ISO-waarde
     \item Gebruik maken van HDR 
   \end{itemize}

Foto’s kunnen ook, net zoals een video, genomen worden aan de hand van de Media capture API. Deze API is echter niet in staat om deze instellingen van de camera aan te passen.

De Image Capture API \autocite{Mandyam2020} is ontwikkeld om meer controle te hebben over de camera. Deze API zorgt ervoor dat instellingen zoals witbalans, temperatuur, exposure, ISO, helderheid, contrast, saturatie, zoom, … programmatisch aangepast kunnen worden.

Deze API heeft standaard geen ondersteuning voor HDR, maar dit kan zelf geïmplementeerd worden aan de hand van ‘third-party-packages’.

\autocite{Bhaumik2019}

\begin{table}[]
	\begin{tabular}{llllll}
		Edge & Firefox & Chrome & Safari & Android (Chrome) & IOS (Safari) \\
		Nee   & Nee      & Ja     & Nee     & Ja               & Nee          
	\end{tabular}	
	\caption{ondersteuning van de Image Capture API}
	\label{ondersteuning van de Image Capture API}
\end{table}



\paragraph{Geluidopname }

De mediarecorder API, \autocite{CasasSanchez2020} door meerdere browsers aangeboden, is een manier om eenvoudig geluidsfragmenten op te nemen en te importeren in een webapplicatie.

Helaas is er voor Apple-toestellen geen ondersteuning. In de toekomst zal deze functie waarschijnlijk ook beschikbaar worden voor deze toestellen. Dit wordt in de volgende versie van Safari (Safari 14) voor desktop verwacht. Voor Safari voor IOS bestaat deze functie al maar is het nog een experimentele functie die de gebruiker zelf moet activeren.


%TODO: checken voor nieuwe update IOS 14
******************** checken voor nieuwe update IOS 14 **************************

\begin{table}[]
	\begin{tabular}{llllll}
		Edge & Firefox & Chrome & Safari & Android (Chrome) & IOS (Safari) \\
		Ja   & Ja      & Ja     & Nee     & Ja               & Nee          
	\end{tabular}	
	\caption{ondersteuning van de Image Capture API}
	\label{ondersteuning van de Image Capture API}
\end{table}

Gelukkig is er een alternatief voorzien met HTML5-tags. Dit is een methode die voor alle platformen zal werken maar niet op dezelfde manier.

\begin{lstlisting}
<input type="file" accept="audio/*" capture>
\end{lstlisting}

Er wordt gebruik gemaakt van een inputveld waar de gebruiker een bestand kan uploaden. Door het accept attribuut wordt duidelijk gemaakt dat enkel audiofragmenten geüpload mogen worden. Het capture attribuut zorgt ervoor dat waar mogelijk de gebruiker een audiofragment kan opnemen in de default geluidsopname app. Dit fragment wordt dan automatisch geïmporteerd in de webapplicatie. Dit is enkel mogelijk op mobiele toestellen en dus niet in desktopbrowsers.

\autocite{Kinlan2019}

Dit is een goed voorbeeld van progressive enhancement. 



\paragraph{Real-time communicatie }

Bij de meeste populaire communicatieapplicaties zoals WhatsApp, Messenger, Skype, … is videobellen mogelijk. Om dit mogelijk te maken moet er live video en audio gestreamd kunnen worden tussen twee of meer personen.

‘Real-time communication in the web’ of WebRTC \autocite{Jennings2020} is een verzameling van API’s die het verzenden en ontvangen van real-time video en audio mogelijk maakt, zonder afhankelijk te zijn van een gecentraliseerde server. Deze server is echter wel nodig om een connectie tot stand te brengen. Eens deze connectie er is, is er een peer-to-peer verbinding.

\begin{table}[]
	\begin{tabular}{llllll}
		Edge & Firefox & Chrome & Safari & Android (Chrome) & IOS (Safari) \\
		Ja   & Ja      & Ja     & Ja     & Ja               & ja          
	\end{tabular}	
	\caption{ondersteuning van WebRTC}
	\label{ondersteuning van WebRTC}
\end{table}




\paragraph{Casting}


Applicaties die media tonen aan de gebruiker kunnen deze casten naar tv-toestel. Dit gebeurt bij Apple toestellen aan de hand van Airplay en bij Android toestellen aan de hand van google cast.


YouTube is een applicatie die hier gebruik van maakt. Als er een video bekeken wordt, zal de gebruiker een optie krijgen om deze te tonen om een tv.


Op Apple toestellen kan een PWA nu ook AirPlay implementeren. 

\autocite{Apple2020a}


De Chrome Sender API \autocite{Developers2020b} zorgt ervoor dat alle modern toestellen media kunnen delen op een tv of ander scherm die dit ondersteund.

\begin{table}[]
	\begin{tabular}{llllll}
		Edge & Firefox & Chrome & Safari & Android (Chrome) & IOS (Safari) \\
		Nee   & Nee      & Nee     & Ja     & Nee               & Ja          
	\end{tabular}	
	\caption{ondersteuning Apple AirPlay}
	\label{ondersteuning Apple AirPlay}
\end{table}
\begin{table}[]
	\begin{tabular}{llllll}
		Edge & Firefox & Chrome & Safari & Android (Chrome) & IOS (Safari) \\
		Ja   & Ja      & Ja     & Ja     & Ja               & ja          
	\end{tabular}	
	\caption{ondersteuning van Chrome Sender API}
	\label{ndersteuning van Chrome Sender API}
\end{table}




\paragraph{Media-controle in de notificatie }


Als een native applicatie media afspeelt op een mobiel toestel, kan deze applicatie bestuurd worden vanuit de notificatie. De afgespeelde media zal ook niet stoppen als een gebruiker de applicatie verlaat

Een voorbeeld hiervan is Spotify, in de notificatie van Spotify kan de gebruiker volgende acties uitvoeren.
 \begin{itemize}
   \item	Informatie bekijken over het nummer
   \item	Naar het volgende nummer gaan
   \item	Het nummer pauzeren
   \item	Het nummer toevoegen aan “mijn favorieten”
   \item	De vooruitgang van het nummer zien en aanpassen
\end{itemize}
De Media Session API \autocite{Beaufort2019} zorgt ervoor dat als er media afgespeeld wordt op een website, en de browser wordt gesloten, de media niet zal stoppen met afspelen.

Deze API zorgt er ook voor dat er een notificatie komt waar de gebruiker controle heeft over de afgespeelde media. 

Het voorbeeld van Spotify kan dus volledig geïmplementeerd worden als PWA.

\begin{table}[]
	\begin{tabular}{llllll}
		Edge & Firefox & Chrome & Safari & Android (Chrome) & IOS (Safari) \\
		Ja   & Ja      & Ja     & Ja     & Ja               & ja          
	\end{tabular}	
	\caption{ondersteuning van Media Session API}
	\label{oondersteuning van Media Session API}
\end{table}



\subsubsection{Connectie met andere apparaten}



\paragraph{Bluetooth }

Native applicaties kunnen een verbinding maken met bluetooth-toestellen. Eens er een verbinding is, kan er informatie uitgewisseld worden tussen de toestellen. Een voorbeeld van een applicatie die hier gebruik van maakt is de ‘Sony Headphones’ app. Aan de hand van deze app kan er verbinding gemaakt worden met een koptelefoon en kunnen de instellingen van de koptelefoon aangepast worden.

Met de Web Bluetooth API \autocite{Grant2020} kan er vanuit de browser verbinding gemaakt worden met bluetooth-toestellen. De web API heeft zowel schrijf- als leesrechten bij externe toestellen. 

Er kan dus geconcludeerd worden dat de Web Bluetooth API kan gebruikt worden voor applicaties die gebruik moeten maken van bluetooth-toestellen.

\autocite{Beaufort2019a}

\begin{table}[]
	\begin{tabular}{llllll}
		Edge & Firefox & Chrome & Safari & Android (Chrome) & IOS (Safari) \\
		Ja   & Nee      & Ja     & Nee     & Ja               & Nee          
	\end{tabular}	
	\caption{ondersteuning van Web Bluetooth API}
	\label{ondersteuning van Web Bluetooth API}
\end{table}



\paragraph{USB}

Verkopers van toestellen met USB kunnen nu gebruik maken van de Web USB API, \autocite{Rockot2020}. Bij het verbinden van een USB-toestel kan er automatisch een website geopend worden waarmee het toestel kan interageren.
 
Dit kan interessant zijn voor toestellen die een eenmalige set-up nodig hebben. Met deze technologie kan vermeden worden dat er overbodige software moet geïnstalleerd worden op het toestel van de gebruiker. 

Dit is echter enkel mogelijk met een beperkt aantal browsers en er moet een HTTPS-verbinding zijn.

\autocite{Beaufort2019b}

\begin{table}[]
	\begin{tabular}{llllll}
		Edge & Firefox & Chrome & Safari & Android (Chrome) & IOS (Safari) \\
		Ja   & Nee      & Ja     & Nee     & Ja               & Nee          
	\end{tabular}	
	\caption{ondersteuning van Web USB API}
	\label{ondersteuning van Web USB API}
\end{table}


\paragraph{NFC}

Near field communication of NFC is een technologie om een kleine hoeveelheid informatie uit te wisselen over een kleine afstand (Maximum 20cm). NFC wordt gebruikt om draadloze betalingen uit te voeren met een betaalkaart of met een smartphone. 
\autocite{Paus2007}

\begin{table}[]
	\begin{tabular}{llllll}
		Edge & Firefox & Chrome & Safari & Android (Chrome) & IOS (Safari) \\
		Nee   & Nee      &  Nee     & Nee     & Nee               & Nee          
	\end{tabular}	
	\caption{ondersteuning van Web NFC API }
	\label{ondersteuning van Web NFC API}
\end{table}

Dit is een functie met veel mogelijkheden die helaas niet beschikbaar is voor webapplicaties.
Er bestaat echter wel een API om gebruik te kunnen maken van NFC \autocite{RohdeChristiansen2020}, maar de Web NFC API is een experimentele API. Dit betekent dat de eindgebruiker dit nog niet kan gebruiken.



\subsubsection{Toestelkenmerken}


\paragraph{Netwerkinformatie}

De Network information API \autocite{Lamouri2020} voorziet informatie over het type netwerkverbinding die de gebruiker momenteel bezit. Deze informatie bevat het connectietype (2g, 3g, 4g) en wat de maximale downloadsnelheid is van deze verbinding.

\begin{table}[]
	\begin{tabular}{llllll}
		Edge & Firefox & Chrome & Safari & Android (Chrome) & IOS (Safari) \\
		Nee   & Nee      &  Ja     & Nee     & Ja               & Nee          
	\end{tabular}	
	\caption{ondersteuning van Network information API }
	\label{ondersteuning van Network information API}
\end{table}

\paragraph{Online status}

dit is een eenvoudige eigenschap die kan opgeroepen worden op het navigator object. Deze eigenschap bevat een booleaanse waarde die “waar” zal zijn als de gebruiker een connectie heeft met het internet. Deze informatie kan interessant zijn bij het ontwikkelen van een PWA met offline functionaliteiten.
\begin{table}[]
	\begin{tabular}{llllll}
		Edge & Firefox & Chrome & Safari & Android (Chrome) & IOS (Safari) \\
		Ja   & Ja      &  Ja     & Ja     & Ja               & Ja          
	\end{tabular}	
	\caption{ondersteuning online status }
	\label{ondersteuning online status}
\end{table}

\paragraph{Vibratiemotor }

De Vibration API \autocite{Kostionen2020} zorgt ervoor dat de vibratiemotor kan aangesproken worden vanuit de webapplicatie.

\begin{table}[]
	\begin{tabular}{llllll}
		Edge & Firefox & Chrome & Safari & Android (Chrome) & IOS (Safari) \\
		Ja   & Ja      &  Ja     & Nee     & Ja               & Nee          
	\end{tabular}	
	\caption{ondersteuning vibratiemotor  }
	\label{ondersteuning vibratiemotor  }
\end{table}


\paragraph{Batterijstatus}

Aan de hand van de Battery Status API \autocite{Kostiainen2020} kan er informatie over de batterij van het toestel verkregen worden.

Volgende informatie kan verkregen worden:
 \begin{itemize}
	\item	Aan het opladen
	\item	Batterijpercentage
	\item	Bij opladen, tijd tot volladen
	\item	Bij niet opladen, tijd tot batterij leeg
\end{itemize}

Aan de hand van deze API kunnen er ook acties uitgevoerd worden op basis van het veranderen van de toestand van de batterij. Er kan bijvoorbeeld een functie uitgevoerd worden als de gebruiker zijn toestel met een energiebron verbindt.

\begin{table}[]
	\begin{tabular}{llllll}
		Edge & Firefox & Chrome & Safari & Android (Chrome) & IOS (Safari) \\
		Ja   & Nee      &  Ja     & Nee     & Ja               & Nee          
	\end{tabular}	
	\caption{ondersteuning batterijstatus  }
	\label{ondersteuning batterijstatus }
\end{table}

\paragraph{Toestelgeheugen}
de Device Memory API \autocite{Panicker2020} geeft informatie over het RAM-geheugen van het toestel van de gebruiker. Dit kan interessant zijn voor het laden van een eventuele lichtere versie van een website voor minder capabele toestellen.

\begin{table}[]
	\begin{tabular}{llllll}
		Edge & Firefox & Chrome & Safari & Android (Chrome) & IOS (Safari) \\
		Ja   & Nee      &  Ja     & Nee     & Ja               & Nee          
	\end{tabular}	
	\caption{ondersteuning toestelgeheugen }
	\label{ondersteuning toestelgeheugen }
\end{table}



\subsubsection{Native gedrag}

\paragraph{Lokale notificaties}
Bij native applicaties kan een bepaalde actie binnen de app resulteren in een notificatie. Veel gezondheids-tracking applicaties maken hier gebruik van. Een gebruiker zal bijvoorbeeld een melding krijgen als een vooropgesteld aantal stappen op een dag is bereikt.

Lokale notificaties zijn beschikbaar via de Notifications API \autocite{Gregg2020}. Lokale notificaties zijn notificaties die geen internet of server nodig hebben. Deze kunnen gepland worden bij het laden van de website. Ze worden dus lokaal geactiveerd.

Meer informatie over notificaties kan gevonden worden in het hoofdstuk ‘functionaliteiten die een servcie worker mogelijk maakt.

Dankzij persistent local notifications en zijn service worker kan het voorbeeld van de fitness-tracking applicatie ook geïmplementeerd worden als PWA.

\begin{table}[]
	\begin{tabular}{llllll}
		Edge & Firefox & Chrome & Safari & Android (Chrome) & IOS (Safari) \\
		Ja   & Nee      &  Ja     & Nee     & Ja               & Nee          
	\end{tabular}	
	\caption{ondersteuning lokale notificaties }
	\label{ondersteuning lokale notificaties}
\end{table}

\paragraph{Push notificaties}

Native applicaties kunnen genieten van notificaties die niet geactiveerd worden vanop het toestel zelf. Een voorbeeld hiervan is een sport-applicatie die een melding geeft als de gebruiker zijn favoriete voetbalploeg een doelpunt heeft gemaakt.

Push notificaties zijn notificaties die verstuurd worden vanop een server. Door gebruik te maken van de Push API \autocite{Sullivan2020} om notificaties te ontvangen en de Notification API om notificaties op het scherm te tonen, kan een PWA push notificaties implementeren. 

Meer informatie over notificaties kan gevonden worden in het hoofdstuk ‘functionaliteiten die een servcie worker mogelijk maakt.

Door deze functionaliteiten kan het voorbeeld van een sportapplicatie ook geïmplementeerd worden als PWA.

\begin{table}[]
	\begin{tabular}{llllll}
		Edge & Firefox & Chrome & Safari & Android (Chrome) & IOS (Safari) \\
		Ja   & Ja      &  Ja     & Nee     & Ja               & Nee          
	\end{tabular}	
	\caption{ondersteuning push notificaties }
	\label{ondersteuning push notificaties}
\end{table}



\paragraph{A2HS}


 Door het toevoegen van een web app manifest kan je de browser duidelijk maken hoe een applicatie er moet uitzien als het toegevoegd wordt aan het startscherm. De PWA zal er dan op het startscherm gelijk uitzien als een native applicatie.
Meer informatie over notificaties kan gevonden worden in het hoofdstuk ‘Wat is een PWA’.


\begin{table}[]
	\begin{tabular}{llllll}
		Edge & Firefox & Chrome & Safari & Android (Chrome) & IOS (Safari) \\
		Ja   & Ja      &  Ja     & Nee     & Ja               & Nee          
	\end{tabular}	
	\caption{ondersteuning A2HS }
	\label{ondersteuning A2HS}
\end{table}


\paragraph{Badges}

Als een native applicatie een melding heeft ontvangen zal er een indicatie staan naast het icoontje op het startscherm. Dit kan nu ook geïmplementeerd worden voor geïnstalleerde PWA’s aan de hand van de Badging API \autocite{LePage2020a}

\begin{table}[]
	\begin{tabular}{llllll}
		Edge & Firefox & Chrome & Safari & Android (Chrome) & IOS (Safari) \\
		Ja   & Onbekend      &  Ja     & Nee     & Ja               & Nee          
	\end{tabular}	
	\caption{ondersteuning badges }
	\label{ondersteuning badges}
\end{table}

\paragraph{Voorgrond-detectie }

Native applicaties kunnen detecteren als een applicatie op de voorgrond wordt gebruikt. YouTube maakt hier gebruik van om zeker te zijn dat de gebruiker de applicatie actief gebruikt op het moment dat een advertentie getoond wordt.

Met de Page Visibility Detection API \autocite{Grigorik2020} kan gedetecteerd worden of een applicatie in de voorgrond gebruikt wordt of niet. 

Aan de hand van deze applicatie kan het gedrag van de applicatie aangepast worden als de gebruiker de applicatie niet meer in de voorgrond gebruikt.

\begin{table}[]
	\begin{tabular}{llllll}
		Edge & Firefox & Chrome & Safari & Android (Chrome) & IOS (Safari) \\
		Ja   & Ja      &  Ja     & Ja     & Ja               & Ja          
	\end{tabular}	
	\caption{ondersteuning voorgrond detectie }
	\label{ondersteuning voorgrond detectie}
\end{table}


\paragraph{Toestemmingen}

Om gebruik te maken van hardware functies van een toestel is vaak, om privacy redenen, de toestemming van de gebruiker nodig. Hiervoor is de Permissions API \autocite{Caceres2020} ontwikkeld. Er kan toestemming gevraagd worden op een gelijkaardige manier voor verschillende functies.

Functies waarvoor toestemming gevraagd kan worden:
 \begin{itemize}
	\item	Locatie
	\item	Notificaties
	\item	Push-notificaties
	\item	Midi (musical instrument digital interface)
	\item	Klembord
	\item	Camera
	\item	Microfoon
	\item	Achtergrondsynchronisatie
	\item	Lichtsensor
	\item	Versnellingsmeter
	\item	Gyroscoop
	\item	Magneetsensor
	\item	Betalingen
\end{itemize}

	\begin{table}[]
		\begin{tabular}{llllll}
			Edge & Firefox & Chrome & Safari & Android (Chrome) & IOS (Safari) \\
			Ja   & Ja      &  Ja     & nee     & Ja               & nee          
		\end{tabular}	
		\caption{ondersteuning Permissions API }
		\label{ondersteuning Permission API}
	\end{table}
	

\subsubsection{Besturingssysteem}
\paragraph{offline opslage}
Native applicaties kunnen nog steeds gebruikt worden als er geen internetverbinding is. Toepassingen die geen netwerkverzoeken doen zijn dus nog volledig operationeel zonder internetverbinding. 

Er zijn verschillende technologieën om data offline op te slaan. 

 \begin{itemize}
	\item	Web storage
	\item	IndexedDB
	\item	Cache API
	\item	Storage API
\end{itemize}

	\subparagraph{Web storage}
	De meest eenvoudige manier om data op te slaan. Er kunnen key-value paren opgeslagen worden in het localStorage of in het sessionStorage. 
	\autocite{Hickson2020}
	
	\begin{table}[]
		\begin{tabular}{llllll}
			Edge & Firefox & Chrome & Safari & Android (Chrome) & IOS (Safari) \\
			Ja   & Ja      &  Ja     & Ja     & Ja               & Ja          
		\end{tabular}	
		\caption{ondersteuning web storrage }
		\label{ondersteuning web storrage}
	\end{table}
	
	
	
	\subparagraph{IndexedDB}
	Een API voor het opslaan van grote hoeveelheden gestructureerde data op het toestel van de eindgebruiker. De data kan snel gelezen worden omdat er indexen gebruikt worden. 
	\autocite{Alabbas2020}
	
	\begin{table}[]
			\begin{tabular}{llllll}
				Edge & Firefox & Chrome & Safari & Android (Chrome) & IOS (Safari) \\
				Ja   & Ja      &  Ja     & Ja     & Ja               & Ja          
			\end{tabular}	
			\caption{ondersteuning IndexedDB}
			\label{ondersteuning IndexedDB}
	\end{table}
	
	
	\subparagraph{Cache API}
	Deze API is gespecialiseerd in het opslaan van netwerkverzoeken. Dit is heel handig in samenwerking met een serviceworker. API-calls kunnen opgeslagen worden voor offline gebruik.
	\autocite{vanKesteren2020}
	
	\begin{table}[]
			\begin{tabular}{llllll}
				Edge & Firefox & Chrome & Safari & Android (Chrome) & IOS (Safari) \\
				Ja   & Ja      &  Ja     & Ja     & Ja               & Nee          
			\end{tabular}	
			\caption{ondersteuning Cache API}
			\label{ondersteuning Cache API}
	\end{table}
	
	
	\subparagraph{Storage API}
	Data die is opgeslagen in een van vorige technologieën kan eenvoudig verwijderd worden door de browser. Met de storage API kan data opgeslagen worden op het systeem voor een langere periode .
	\autocite{Mozilla2020b}
	
	\begin{table}[]
		\begin{tabular}{llllll}
			Edge & Firefox & Chrome & Safari & Android (Chrome) & IOS (Safari) \\
			Ja   & Ja      &  Ja     & Ja     & Ja               & Nee          
		\end{tabular}	
		\caption{ondersteuning Storage API}
		\label{ondersteuning Storage API}
	\end{table}
	
	
	
Door gebruik te maken van deze verschillende API’s kan er ook een offline ervaring aangeboden worden aan de gebruiker. 
	
\paragraph{Bestandentoegang}

Native applicaties hebben toegang tot het volledige bestandssysteem van het toestel. Er kunnen bestaande bestanden gelezen en aangepast worden. Er kunnen ook nieuwe bestanden aangemaakt en opgeslagen worden.

Door gebruik te maken van de File API \autocite{Kruisselbrink2020} heeft een webapplicatie ook toegang tot het bestandssysteem. Bestanden kunnen gelezen worden en metadata over deze bestanden kan verkregen worden.

Een webapplicatie heeft echter enkel leesrechten op deze bestanden. Er kunnen dus geen bestanden geschreven of aangepast worden.

Dit betekent dus dat bepaalde toepassingen die hier gebruik van maken nog steeds een native applicatie nodig hebben.

\begin{table}[]
	\begin{tabular}{llllll}
		Edge & Firefox & Chrome & Safari & Android (Chrome) & IOS (Safari) \\
		Ja   & Ja      &  Ja     & Nee     & Ja               & Nee          
	\end{tabular}	
	\caption{ondersteuning File API}
	\label{ondersteuning File API}
\end{table}	


\paragraph{Contacten}
Bepaalde native applicaties hebben toegang nodig tot de contacten van de gebruiker. Een voorbeeld hiervan is WhatsApp. Deze importeert de contacten van het toestel in de applicatie.

De contacten die opgeslagen staan op het systeem van de gebruiker kunnen geïmporteerd worden in een webapplicatie met de Contacts API \autocite{Tibbett2020}.

In theorie zouden PWA’s hier dus gebruik kunnen van maken maar de ondersteuning is heel beperkt. Het is dus niet aangeraden om een webapplicatie te ontwikkelen die afhankelijk is van de contacten van een gebruiker.

\begin{table}[]
	\begin{tabular}{llllll}
		Edge & Firefox & Chrome & Safari & Android (Chrome) & IOS (Safari) \\
		Nee   & Nee      &  Nee     & Nee     & Beperkt               & Nee          
	\end{tabular}	
	\caption{ondersteuning Contacts API - * Op het moment van schrijven is dit een nieuwe en experimentele functie die enkel werkt op 
	Android 10. Verdere ondersteuning is nog onbekend.
	}
	\label{ondersteuning Contacts API}
\end{table}	


\paragraph{Sms}
Native Applicaties kunnen binnenkomende sms-berichten lezen. Dit wordt vaak gebruikt om de authenticatie van een gebruiker sneller te laten verlopen. Een voorbeeld van deze use-case kan bij de applicatie van het betalingsplatform PayPal gevonden worden. Als een gebruiker zich registreert, zal zijn telefoonnummer gecontroleerd worden door er een sms naar dit nummer te sturen met een code. PayPal zal zien dat er een sms binnenkomt en zal automatisch de code uit dit bericht halen. Op deze manier hoeft de gebruiker de app niet te verlaten.

Native applicaties kunnen niet enkel sms’en lezen, ze kunnen er ook schrijven. Dit betekent dus dat elke ontwikkelaar een sms-client applicatie kan maken.

Met de SMS-receiver API \autocite{Fullea2020} kan er gekeken worden naar inkomende sms’en. Het voorbeeld van de PayPal applicatie kan dus ook geïmplementeerd worden als PWA. PWA’s hebben echter enkel toegang tot binnenkomende sms’en. 

Het voorbeeld van PayPal kan ook geïmplementeerd worden aan de hand van een PWA. Applicaties die ook sms-berichten moeten versturen kunnen niet ontwikkeld worden als PWA.

\begin{table}[]
	\begin{tabular}{llllll}
		Edge & Firefox & Chrome & Safari & Android (Chrome) & IOS (Safari) \\
		Nee   & Nee      &  Nee     & Nee     & Beperkt               & Nee          
	\end{tabular}	
	\caption{ondersteuning Messaging API - * Op het moment van schrijven is dit een nieuwe en experimentele functie die enkel werkt op 
	Android 10. Verdere ondersteuning is nog onbekend.
	}
	\label{ondersteuning Messaging API}
\end{table}	



\paragraph{Taakplanning}
De Task Sheduler API \autocite{Kulkarni2020} kan ervoor zorgen dat taken zoals alarmen, herinneringen en gelijkaardige taken kunnen ingepland worden in het systeem. Deze API is slechts een voorstel en heeft dus nog geen ondersteuning.

Een applicatie schrijven die het alarm van een smartphone in de ochtend laat afgaan, of een activiteit in jouw agenda plaatst, is dus niet mogelijk met een PWA. Dit zijn toepassingen die wel mogelijk zijn met native applicaties.

\begin{table}[]
	\begin{tabular}{llllll}
		Edge & Firefox & Chrome & Safari & Android (Chrome) & IOS (Safari) \\
		Nee   & Nee      &  Nee     & Nee     & Nee               & Nee          
	\end{tabular}	
	\caption{ondersteuning Task Sheduler API }
	\label{ondersteuning Task Sheduler API}
\end{table}	



\subsubsection{Input}

\paragraph{Touch gebaren}

Native applicaties hebben een verwachtingspatroon ontwikkeld bij de gebruiker. Voorbeelden hiervan zijn:

 \begin{itemize}
	\item	Swipe van links opent het menu
	\item	Knijpen om in te zoomen
\end{itemize}

HTML5 voegt aan de reeds bestaande input methodes nu ook touch-controls toe. Dit is belangrijk om een applicatie intuïtief te laten werken. Het is logisch dat Safari op desktop dit niet ondersteunt aangezien Safari enkel kan gedownload worden op Mac-toestellen en geen enkel Mac-toestel een touchscreen heeft.

Deze gebaren kunnen nu ook gebruikt worden in een PWA. Dit zorgt ervoor dat een geïnstalleerde PWA meer zal aanvoelen als een native applicatie.

\begin{table}[]
	\begin{tabular}{llllll}
		Edge & Firefox & Chrome & Safari & Android (Chrome) & IOS (Safari) \\
		Ja   & Ja      &  Ja     & Ja     & ja               & Ja          
	\end{tabular}	
	\caption{ondersteuning touch gebaren}
	\label{ondersteuning touch gebaren}
\end{table}	

\paragraph{Klembord toegang}
De Clipboard API \autocite{Kacmarcik2020} geeft een ontwikkelaar de mogelijkheid om te interageren met het klembord. Er kunnen zowel items van het klembord gelezen worden als dat er items kunnen geschreven worden naar het klembord.

Er worden ook methodes voorzien voor het reageren op de actie waarbij een gebruiker zelf iets kopieert of plakt. 

\begin{table}[]
	\begin{tabular}{llllll}
		Edge & Firefox & Chrome & Safari & Android (Chrome) & IOS (Safari) \\
		Ja   & Ja      &  Ja     & Ja     & ja               & Ja          
	\end{tabular}	
	\caption{ondersteuning Clipboard API}
	\label{ondersteuning Clipboard API}
\end{table}	




\subsubsection{User experience}
\subsubsection{Locatie en positionering}
\subsubsection{Scherm en output}

\subsection{Conclusie }

\subsubsection{Browsers}

\subsubsection{Mobiele besturingssystemen }

\subsubsection{Desktop besturingssystemen}