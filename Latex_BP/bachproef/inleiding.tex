%%=============================================================================
%% Inleiding
%%=============================================================================

\chapter{Inleiding}
\label{ch:inleiding}

In deze thesis zal er onderzoek gedaan woorden naar Progressive Web applications (PWA).

Een PWA is een webapplicatie die gebruik maken van moderne webtechnologieën om op deze manier een ervaring aan te bieden die dicht die van een native applicatie ligt.

\section{Probleemstelling}

	Het uitbrengen van een digitaal product is vaak duur en tijdrovend. Dit komt omdat er voor meerdere platformen vaak verschillende codebases nodig zijn en er dus duplicate code geschreven moet worden. 
	
	PWA's kunnen een oplossing bieden voor bedrijven die zo snel mogelijk een applicatie willen publiceren op zoveel mogelijke platformen.
	
	Voor digitale agentschappen en start-ups is het vaak essentieel om een product naar buiten te kunnen brengen in een zo kort mogelijke tijd om een competitief voordeel de creëren. 
	
	Het web kan een oplossing bieden voor beide problemen, veel developers zijn vaardig met webtechnologieën en een webapplicatie kan op alle mogelijke moderne toestellen gebruikt worden. 
	
	Echter heeft het web al altijd een aantal beperkingen gehad, een van de doelen van PWA's is deze gebreken overkomen.


\section{Onderzoeksvraag}

	% TODO : Herschrijven onderzoekvragen

	Deze thesis zal een antwoord bieden op de vraag: "Welke applicaties kunnen er ontwikkeld worden als progressive web application?".
	
	Om tot on antwoord  te kunnen formulieren op deze onderzoeksvraag zal er eerst antwoord moeten geboden worden op volgende onderzoeksvragen:
	
	\begin{itemize}
		  \item Welke stappen zijn nodig om een traditionele website om te vormen tot een PWA?
		  \item Wat zijn de beperkingen van een PWA?
		  \item Kan een PWA alle functionaliteiten gebruiken die beschikbaar zijn voor native applications?
		  \item Hoe staan de verschillende besturingssystemen ten opzichte van PWA's?
		  \item Welke andere technologieën kunnen er gebruikt worden om applicaties te ontwikkelen voor meerdere platformen waarbij er maar één codebase is?
	  \end{itemize}

\section{Onderzoeksdoelstelling}
	
	Om deze onderzoeksvragen goed te kunnen beantwoorden zullen er verschillende onderdelen in deze bachelorproef uitgewerkt worden.
	
	Er zal een literatuurstudie uitgevoerd worden, deze zal een beeld schetsen van wat er functioneel wel en niet mogelijk is met PWA's. 
	De literatuurstudie zal een antwoord bieden op volgende onderzoeksvragen:
	\begin{itemize}
		  \item Wat zijn de beperkingen van een PWA?
		  \item Kan een PWA alle functionaliteiten gebruiken die beschikbaar zijn voor native applications?
		  \item Hoe staan de verschillende besturingssystemen ten opzichte van PWA's?
		  \item Welke andere technologieën kunnen er gebruikt worden om applicaties te ontwikkelen voor meerdere platformen waarbij er maar één codebase is?
	  \end{itemize}	
	  
	Na de literatuurstudie zal er onderzocht worden welke stappen er moeten ondernomen worden om van een traditionele website een PWA te maken. Dit zal gebeuren aan de hand van een voorbeeld.
	
	Vervolgens zal er een proof-of-concept ontwikkeld worden, deze zal een aantal functionaliteiten die beschikbaar zijn voor PWA's testen. Er zal een nadruk liggen op de gebruikerservaring die aangeboden kan worden met PWA's.
	
	Op basis van de literatuurstudie en de ondervindingen van de proof-of-concept zal er een besluit geformuleerd worden.
	
\section{Opzet van deze bachelorproef}

	In Hoofdstuk~\ref{ch:stand-van-zaken} wordt een overzicht gegeven van de stand van zaken binnen het onderzoeksdomein, op basis van een literatuurstudie.
	
	In Hoofdstuk~\ref{ch:methodologie} wordt de methodologie toegelicht en worden de gebruikte onderzoekstechnieken besproken om een antwoord te kunnen formuleren op de onderzoeksvragen.
	
	In Hoofdstuk ~\ref{ch:TransformerenNaarEenPWA} zal er onderzocht worden welke stappen er ondernomen moeten worden om een traditionele website toe te kunnen voegen aan het startscherm, en dus om te vormen tot een PWA.
	
	in Hoofdstuk ~\ref{ch:Proof-of-concept} zal er een proof-of-conept uitgewerkt die bekijkt welke functionaliteiten er beschikbaar zijn voor PWA's en welke gebruikservaring er aangeboden kan worden.
	
	In Hoofdstuk~\ref{ch:conclusie}, tenslotte, wordt de conclusie gegeven en een antwoord geformuleerd op de onderzoeksvragen. Daarbij wordt ook een aanzet gegeven voor toekomstig onderzoek binnen dit domein.
	




%De inleiding moet de lezer net genoeg informatie verschaffen om het onderwerp te begrijpen en in te zien waarom de onderzoeksvraag de moeite waard is om te onderzoeken. In de inleiding ga je literatuurverwijzingen beperken, zodat de tekst vlot leesbaar blijft. Je kan de inleiding verder onderverdelen in secties als dit de tekst verduidelijkt. Zaken die aan bod kunnen komen in de inleiding~\autocite{Pollefliet2011}:

%\begin{itemize}
 % \item context, achtergrond
  %\item afbakenen van het onderwerp
 % \item verantwoording van het onderwerp, methodologie
%  \item probleemstelling
%  \item onderzoeksdoelstelling
%  \item onderzoeksvraag
%  \item \ldots
%\end{itemize}

%\section{\IfLanguageName{dutch}{Probleemstelling}{Problem Statement}}
%\label{sec:probleemstelling}

%Uit je probleemstelling moet duidelijk zijn dat je onderzoek een meerwaarde heeft voor een concrete doelgroep. De doelgroep moet goed gedefinieerd en afgelijnd zijn. Doelgroepen als ``bedrijven,'' ``KMO's,'' systeembeheerders, enz.~zijn nog te vaag. Als je een lijstje kan maken van de personen/organisaties die een meerwaarde zullen vinden in deze bachelorproef (dit is eigenlijk je steekproefkader), dan is dat een indicatie dat de doelgroep goed gedefinieerd is. Dit kan een enkel bedrijf zijn of zelfs één persoon (je co-promotor/opdrachtgever).

%\section{\IfLanguageName{dutch}{Onderzoeksvraag}{Research question}}
%\label{sec:onderzoeksvraag}

%Wees zo concreet mogelijk bij het formuleren van je onderzoeksvraag. Een onderzoeksvraag is trouwens iets waar nog niemand op dit moment een antwoord heeft (voor zover je kan nagaan). Het opzoeken van bestaande informatie (bv. ``welke tools bestaan er voor deze toepassing?'') is dus geen onderzoeksvraag. Je kan de onderzoeksvraag verder specifiëren in deelvragen. Bv.~als je onderzoek gaat over performantiemetingen, dan 

%\section{\IfLanguageName{dutch}{Onderzoeksdoelstelling}{Research objective}}
%\label{sec:onderzoeksdoelstelling}

%Wat is het beoogde resultaat van je bachelorproef? Wat zijn de criteria voor succes? Beschrijf die zo concreet mogelijk. Gaat het bv. om een proof-of-concept, een prototype, een verslag met aanbevelingen, een vergelijkende studie, enz.

%\section{\IfLanguageName{dutch}{Opzet van deze bachelorproef}{Structure of this bachelor thesis}}
%\label{sec:opzet-bachelorproef}

% Het is gebruikelijk aan het einde van de inleiding een overzicht te
% geven van de opbouw van de rest van de tekst. Deze sectie bevat al een aanzet
% die je kan aanvullen/aanpassen in functie van je eigen tekst.

%De rest van deze bachelorproef is als volgt opgebouwd:

%In Hoofdstuk~\ref{ch:stand-van-zaken} wordt een overzicht gegeven van de stand van zaken binnen het onderzoeksdomein, op basis van een literatuurstudie.

%In Hoofdstuk~\ref{ch:methodologie} wordt de methodologie toegelicht en worden de gebruikte onderzoekstechnieken besproken om een antwoord te kunnen formuleren op de onderzoeksvragen.

% TODO: Vul hier aan voor je eigen hoofstukken, één of twee zinnen per hoofdstuk

%In Hoofdstuk~\ref{ch:conclusie}, tenslotte, wordt de conclusie gegeven en een antwoord geformuleerd op de onderzoeksvragen. Daarbij wordt ook een aanzet gegeven voor toekomstig onderzoek binnen dit domein.