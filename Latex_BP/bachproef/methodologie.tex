%%=============================================================================
%% Methodologie
%%=============================================================================

\chapter{Methodologie}
\label{ch:methodologie}

%% TODO: Hoe ben je te werk gegaan? Verdeel je onderzoek in grote fasen, en
%% licht in elke fase toe welke stappen je gevolgd hebt. Verantwoord waarom je
%% op deze manier te werk gegaan bent. Je moet kunnen aantonen dat je de best
%% mogelijke manier toegepast hebt om een antwoord te vinden op de
%% onderzoeksvraag.
Hoofdstuk 2 was niet enkel onderbouwd aan de hand van de traditionele bronnen van een literatuurstudie. 
Er werd ook waardevolle informatie verzameld door een interview af te nemen met Thomas Steiner en Wassim Chegham. In deze interviews werd nogmaals duidelijk dat PWA's een heel sterk concept zijn dat relevant zal blijven de komende jaren binnen de developer community.

%todo ✅ refereren naar de appendix 
%todo ✅ en de voornaamste conclusie van de interviews samenvatten
%todo ❌ zijn er zaken van het interview dat ik wil aantonen in mijn POC's  - nee
Thomas Steiner is een developer advocate bij Google. Hij schrijft regelmatig artikels die gepubliceerd worden op het \href{https://web.dev/authors/thomassteiner/}{web.dev} platform. Deze artikels gaan vaak over moderne web-API's en PWA's. Ook publiceerde hij reeds verschillende artikels op \href{https://medium.com/@tomayac}{Medium}.

Wassim Chegham is een google developer expert en is lid van het Angular team. Momenteel is hij werkzaam bij Microsoft als senior cloud advocate. Hij ontwikkeldt ook mee aan verschillende open source projecten. Hij was een google developer expert op het moment dat Google voor het eerst de term PWA gebruikte. Hij gebruikt en onderzoekt PWA's sinds dan.

In beide interviews werden verschillende voordelen aangehaald.  Zowel de ontwikkelkost als het de onafhankelijkheid van app-stores kwamen in beide interviews naar boven. Er werd ook gevraagd naar de nadelen van PWA's. het grootste probleem met PWA's is de verschillende en  inconsistente ondersteuning van de browsers.
Wassim Chegham maakte ook duidelijk dat PWA's niet enkel een technologie zijn die voor snellere en goedkopere productie kan zorgen. PWA's zijn een unieke technologie die ook andere strategische voordelen heeft voor een organisatie. Het is een grote troef dat een applicatie kan gebruikt worden zonder geïnstalleerd te moeten worden zoals bij een native applicatie. Volgens hem is het op die manier makkelijker om gebruikers te winnen.
De conclusie van beide ontwikkelaars was dat PWA's een technologie zijn die voor elk project overwogen moet worden. Beide zijn ook overtuigd dat PWA's steeds relevanter zullen worden en niet snel zullen verdwijnen.

De volledige interviews kunnen in de appendix A gevonden worden.

Na het voeren van de brede literatuurstudie in hoofdstuk 2 zullen er 2 cases uitgewerkt worden die meer inzicht zullen geven in de technologie.

% todo ✅ welke conversion rate - engagement er bekeken zal worden (is dit niet engagement ipv conversion)
%todo ❌ Hoe ga je de engagement/ conversion testen
%todo ✅ Wat bedoel je met in kaart brangen wat nodig is om dit te bereiken?
%todo ✅ Hier mag je al spreken van de lighthouse audit en de criteria die voor een pwa belangrijk zijn.
%todo ❌ OZT toevoegen voor snelheid pagina te vergelijken

Eerst zal er een bestaande webapplicatie omgevormd worden tot een PWA die installeerbaar is en die offline gebruikt kan worden.

Dit is een belangrijk onderzoek omdat het aanwezig zijn op het startscherm tot een betere engagement kan leiden. Een gebruiker die een applicatie heeft toegevoegd aan zijn startscherm zal gemiddeld meer en langere sessies hebben dan bij een traditionele webapplicatie.
 \autocite{LePage2020b}

Door concreet te documenteren welke stappen er nodig zijn zijn om een traditionele web applicatie om te vormen tot een PWA kunnen ontwikkelaars en organisaties beter inschatten hoelang dit zal duren.
Op basis van deze inschatting kunnen ze een betere beslissing maken als ze dit willen implementeren of niet.

Om een website te kunnen installeren op een toestel moet er voldaan worden aan drie voorwaarden:
\begin{itemize}
	\item Een HTTPS-connectie hebben
	\item Een app manifest hebben
	\item Een service worker registreren
\end{itemize}

Dit zal gecontroleerd worden aan de hand van een lighthouse audit.
	
%todo: Hoe ga je de OV onderzoeken
%todo: Welke onderzoekstechnieken worden er gebruikt
%todo: Verschillende browsers testen
%todo: Wat is native waardig - welke criteria

Vervolgens zal er ook een proof-of-concept uitgewerkt worden waarin moderne web-technologieën gebruikt worden om een PWA te ontwikkelen met een video-bel functionaliteit. In deze proof of concept zal onderzocht worden als een PWA een native waardige ervaring kan bieden. Hier zullen concepten de concepten besproken in hoofdstuk 2 toegepast worden, voorbeelden hiervan zijn de application shell en  progressive enhancement.





