%%=============================================================================
%% Methodologie
%%=============================================================================

\chapter{Methodologie}
\label{ch:methodologie}

%% TODO: Hoe ben je te werk gegaan? Verdeel je onderzoek in grote fasen, en
%% licht in elke fase toe welke stappen je gevolgd hebt. Verantwoord waarom je
%% op deze manier te werk gegaan bent. Je moet kunnen aantonen dat je de best
%% mogelijke manier toegepast hebt om een antwoord te vinden op de
%% onderzoeksvraag.

De kennis die verworven werd om hoofdstuk 2 te schrijven kwam van verschillende bronnen. 
Hierbij waren de vorstelijkste bronnen de ontwikkelaarswebsite van browsers. Voorbeelden hiervan zijn \href{https://developers.google.com/}{Google Developers} en \href{https://developer.mozilla.org/en-US/}{MDN web docs}.

Ook werden er scripties van andere onderwijsinstellingen geraadpleegd om meer diepgaande informatie te vergaren.

PWA's zijn een populair gespreksonderwerp binnen de web development community. Dit vertaalt zicht in het aantal talks er gegeven worden over het onderwerp. Ook deze conference talks werden gebruikt als bron voor het schrijven van deze scriptie. Deze trend vertaalde zich ook in het aantal artikels er gepubliceerd worden op platformen zoal Medium.

Er werd ook waardevolle informatie verzameld door een interview af te nemen met Thomas Steiner en Wassim Chegham. Hier werd duidelijk dat PWA's een begrip is dat meer een meer impact zal hebben, ook gaven ze goede argumenten waarom PWA's geen niet het zoveelste javascript framework of technologie is die veel aandacht krijgt in het begin maar die uiteindelijk weinig geïmplementeerd wordt.


Thomas Steiner is een developer advocate bij Google. Hij schrijft regelmatig artikels die gepubliceerd worden op het \href{https://web.dev/authors/thomassteiner/}{web.dev} platform. Deze artikels gaan vaak over moderne web-API's en PWA's. OOk publiceerde hij reeds verschillende artikels op \href{https://medium.com/@tomayac}{Medium}

Wassim Chegham is een google developer expert en is lid van het Angular team. Momenteel is hij werkzaam bij Microsoft als senior cloud advocate. Hij ontwikkeld ook mee aan verschillende open source projecten. Hij was een google developer expert op het moment dat Google voor het eerst de term PWA gebruikte. Hij gebruikt en onderzoekt PWA's sinds dan.

Na het voeren van de brede literatuurstudie in hoofdstuk 2 zullen er praktisch 2 cases uitgewerkt worden die meer inzicht zullen geven in de technologie.

Eerst zal er een bestaande webapplicatie omgevormd worden tot een PWA die installeerbaar is en die offline gebruikt kan worden. Dit is een belangrijk onderzoek omdat het aanwezig zijn op het startscherm tot een betere conversion rate kan leiden. Door het concreet in kaart brengen van wat nodig is om dit te bereiken kunnen ontwikkelaars en  instanties beslissen als ze de tijd willen vrij maken om dit te doen voor hun volgende webapplicatie.

Vervolgens zal er ook een proof-of-concept uitgewerkt worden waarin moderne web-technologieën gebruikt worden om een PWA te ontwikkelen met een video-bel functionaliteit. In deze proof of concept zal onderzocht worden als een PWA een native waardige ervaring kan bieden. Hier zullen concepten de concepten besproken in hoofdstuk 2 toegepast worden, voorbeelden hiervan zijn de application shell en  progressive enhancement.

